%%%%%%%%%%%%%%%%%%%%%%%%%%%%%%%%%%%%%%%%%%%%%%%%%%%%%%%%%%%%%%%%%%%%%%%%%%%%%%%
% Projeto de pesquisa para concurso público de Professor Doutor na USP.
%
% Formatação inspirada em:
% * https://tug.org/pracjourn/2008-1/mori/mori.pdf
% * https://github.com/santisoler/phd-thesis
% * https://github.com/compgeolab/dissertation-template
%%%%%%%%%%%%%%%%%%%%%%%%%%%%%%%%%%%%%%%%%%%%%%%%%%%%%%%%%%%%%%%%%%%%%%%%%%%%%%%

%%%%%%%%%%%%%%%%%%%%%%%%%%%%%%%%%%%%%%%%%%%%%%%%%%%%%%%%%%%%%%%%%%%%%%%%%%%%%%%
% Set a class and import packages
\documentclass[11pt,a4paper,oneside]{book}

% Variables
\newcommand{\Year}{2023}
\newcommand{\Author}{Leonardo Uieda}
\newcommand{\Title}{Projeto de pesquisa de \Author{} para concurso público - Professor Doutor em Geofísica - IAG/USP}
\newcommand{\ORCID}{0000-0001-6123-9515}

\usepackage[utf8]{inputenc}
\usepackage[T1]{fontenc}
\usepackage[brazil]{babel}
\usepackage{geometry}
\usepackage{graphicx}
\usepackage{amssymb}
\usepackage{amsmath}
\usepackage{hyperref}
% create fancy headers
\usepackage{fancyhdr}
% commands for managing dates and its formats
\usepackage{datetime2}
% improved urls with proper hyphenation
\usepackage{xurl}
% Import enumitem to customize descriptions in license.tex
\usepackage{enumitem}
% Tweak the look of captions
\usepackage{caption}
% To control the style of section titles
\usepackage{titlesec}
% Add the bibliography to the table of contents
\usepackage[nottoc,chapter]{tocbibind}
\usepackage[round,authoryear,sort]{natbib}
% show dois as links on references
\usepackage{doi}
% Icon fonts (requires using xelatex or luatex)
\usepackage{fontawesome5}
\usepackage{academicons}
\usepackage{fontspec}
% Set fonts (requires compilation with xelatex)
\usepackage{fontspec}
\usepackage{microtype}
% To make fancy text boxes
\usepackage{xcolor}
\usepackage[framemethod=tikz]{mdframed}
% For fancy and multipage tables
\usepackage{tabularx}
\usepackage{ltablex}
% To define custom environments
\usepackage{environ}
\usepackage{setspace}
% Reference sections by name
\usepackage{nameref}
% Better handling of footnotes inside summary boxes
\usepackage{footmisc}
%%%%%%%%%%%%%%%%%%%%%%%%%%%%%%%%%%%%%%%%%%%%%%%%%%%%%%%%%%%%%%%%%%%%%%%%%%%%%%%

%%%%%%%%%%%%%%%%%%%%%%%%%%%%%%%%%%%%%%%%%%%%%%%%%%%%%%%%%%%%%%%%%%%%%%%%%%%%%%%
% Configuration of the document

\geometry{%
  left=30mm,
  right=30mm,
  top=20mm,
  bottom=15mm,
  headsep=5mm,
  headheight=5mm,
  footskip=10mm,
  includehead=true,
  includefoot=true
}

\setmainfont[%
  Path=fonts/notoserif/,
  UprightFont=NotoSerif-Regular,
  BoldFont=NotoSerif-Bold,
  ItalicFont=NotoSerif-Italic,
  Extension=.ttf
]{NotoSerif}

% Increase the line spacing
\SetSinglespace{1.2}
\doublespacing

% Add a link to a DOI
\newcommand{\DOI}[1]{\url{https://doi.org/#1}}

% Define custom colors
\definecolor{lu_gray}{gray}{0.98}
\definecolor{lu_darkgray}{gray}{0.3}
\definecolor{lu_blue}{RGB}{32, 96, 194}
\definecolor{lu_lightblue}{RGB}{238, 245, 250}
\definecolor{lu_yellow}{RGB}{255, 193, 7}
\definecolor{lu_lightyellow}{RGB}{255, 249, 230}

% Customize how Chapter headings are displayed
\titleformat{\chapter}[display]{\normalfont}{\large Parte \thechapter}{0pt}{\huge}[\titlerule]
\titlespacing*{\chapter}{0pt}{-40pt}{40pt}

% Set the spacing between bibliography entries (requires natbib)
\setlength{\bibsep}{0pt}

% Configure captions
\captionsetup{labelfont=bf,font={color=lu_darkgray},skip=0pt}

% Define a fancy text box
\mdfdefinestyle{summarybox}{%
  leftline=true,
  rightline=false,
  topline=false,
  bottomline=false,
  linewidth=4pt,
  linecolor=lu_blue,
  frametitlefont=\bfseries\color{black}\small,
  frametitlebackgroundcolor=lu_lightblue,
  frametitleaboveskip=7pt,
  frametitlebelowskip=7pt,
  frametitlerule=true,
  frametitlerulewidth=1pt,
  backgroundcolor=lu_gray,
  innertopmargin=7pt,
  innerbottommargin=10pt,
  innerleftmargin=15pt,
  innerrightmargin=15pt,
}
\newmdenv[style=summarybox]{summarybox}
\mdfdefinestyle{subsummarybox}{%
  leftline=true,
  rightline=false,
  topline=false,
  bottomline=false,
  linewidth=4pt,
  linecolor=lu_yellow,
  frametitlefont=\bfseries\color{black}\small,
  frametitlebackgroundcolor=lu_lightyellow,
  frametitleaboveskip=7pt,
  frametitlebelowskip=7pt,
  frametitlerule=true,
  frametitlerulewidth=1pt,
  backgroundcolor=lu_gray,
  innertopmargin=7pt,
  innerbottommargin=10pt,
  innerleftmargin=15pt,
  innerrightmargin=15pt,
}
\newmdenv[style=subsummarybox]{subsummarybox}

% Define something like an fa-ul and a date list
\NewEnviron{fa-ul}{%
  \vspace{-0.4cm}
  \small
  \renewcommand{\arraystretch}{1.25}
  \begin{tabularx}{\linewidth}{@{}p{0.05\linewidth}@{}@{}p{0.95\linewidth}@{}}
    \BODY
  \end{tabularx}%
}

% Configure hyperref and add PDF metadata
\hypersetup{
    colorlinks,
    allcolors=lu_blue,
    pdftitle={\Title},
    pdfauthor={\Author},
    pdftex,
    breaklinks=true,
}

% make urls use the same font as every other text
\urlstyle{same}

% Prevent footnotes from being broken into multiple pages
\interfootnotelinepenalty=10000

% Configure headers and footers
\fancyhf{}
\lhead{\fontsize{9pt}{0}\selectfont\itshape \nouppercase\leftmark}
\chead{}
\rhead{\fontsize{9pt}{0}\selectfont \thepage}
\cfoot{}
\renewcommand{\headrulewidth}{0.3pt}
\renewcommand{\chaptermark}[1]{\markboth{\MakeUppercase{#1}}{}}
%%%%%%%%%%%%%%%%%%%%%%%%%%%%%%%%%%%%%%%%%%%%%%%%%%%%%%%%%%%%%%%%%%%%%%%%%%%%%%%

%%%%%%%%%%%%%%%%%%%%%%%%%%%%%%%%%%%%%%%%%%%%%%%%%%%%%%%%%%%%%%%%%%%%%%%%%%%%%%%
\begin{document}

\pagestyle{plain}
\frontmatter

\begin{titlepage}
  \begin{center}
    \includegraphics[height=2cm]{images/logo.pdf}
    \vspace{1cm}

    PROJETO DE PESQUISA PARA CONCURSO PÚBLICO

    PROFESSOR DOUTOR (RDIDP) EM MÉTODOS POTENCIAIS

    UNIVERSIDADE DE SÃO PAULO
    \vspace{5cm}

    \textbf{\LARGE PROJETO DE PESQUISA}
    \vspace{1cm}

    \textbf{\LARGE \MakeUppercase{\Author{}}}
    \vspace{5cm}

    {\small
      Apresentado para concurso público de títulos e provas para cargo de

      Professor Doutor junto ao Departamento de Geofísica do

      Instituto de Astronomia, Geofísica e Ciências Atmosféricas da

      Universidade de São Paulo.
      \vspace{1cm}

      Edital ATAc-IAG/044/2022
    }
    \vfill

    \Year{}
  \end{center}
\end{titlepage}

\tableofcontents

\mainmatter
\pagestyle{fancy}

%==============================================================================
\chapter{Introdução}

Resumo geral.
Como o projeto está organizado.
X temas que serão explorados abaixo.
O que liga esses temas.

Possuo Bacharelado em Geofísica pela Universidade de São Paulo e Mestrado e Doutorado
em Geofísica pelo Observatório Nacional. Ao longo da minha formação e carreira, pas-
sei por seis instituições de ensino superior em quarto países diferentes. Trabalhei como
Professor Assistente na Universidade do Estado do Rio de Janeiro, Professor Visitante na
University of Hawai‘i at Mānoa e atualmente sou Lecturer (equivalente a Professor Dou-
tor) na University of Liverpool. Ministrei 10 disciplinas diferentes a nível de graduação e
17 cursos de curta duração abrangendo uma gama de tópicos da geofísica, geologia e pro-
gramação. Sou autor de 16 artigos científicos que agregam mais de 1700 citações1 . Atuo na
área de ciência aberta e reprodutibilidade desde meu primeiro artigo publicado em 2012
durante meu Mestrado. Desenvolvo diversos projetos de software livre para ciência que
atraem centenas de milhares de downloads mensais2 . Sou reconhecido por minha exper-
tise em Geociências, desenvolvimento de software e ciência aberta, tendo servido como
editor do Journal of Open Source Software e na coorde

\section{Objetivos principais}

\section{Impacto da pesquisa}



%==============================================================================
\chapter{Modelagem de dados de microscopia magnética}

\begin{summarybox}[frametitle=\faInfoCircle{}\quad Informações principais]
  \begin{fa-ul}
    \faSearchDollar & Possíveis fontes de fianciamento: \\
    \faUserGraduate & Involvimento de alunos: \\
    \faUsers & Colaborações:
  \end{fa-ul}
\end{summarybox}

\section{Preâmbulo}

\section{Objetivos principais}

\section{Introdução}

\section{Work packages}

\newpage

\section{Fontes de financiamento}

\section{Impacto da pesquisa}

%==============================================================================
\backmatter
\bibliographystyle{apalike-doi}
\bibliography{references}

\end{document}
