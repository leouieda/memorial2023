\chapter{Introdução}

Resumo dos títulos e entrada na UERJ.

Ciência aberta e código no github e site (coisas na net).

Papers: quantos, colaborações.

Trabalhos em congressos: quantos, colaborações, quais congressos.

Como vou organizar esse memorial.
Divido em três temas.

O primeiro, Software, está por trás de todo meu trabalho nos demais.
Desenvolvo esses programas para facilitar minha pesquisa, complementar e
fortalecer minhas técnicas de ensino e como forma de conectar com a comunidade
da geofísica nacional e internacional.

Tive dificuldade em elaborar uma organização coerente para esses temas.
O desenvolvimento de software esteve presente ao longo de toda minha carreira.
Uma ordem cronológica seria muito confuso por que os temas se misturam contribuem entre si.

Software
* Sobre como eu me envolvi com programação e software livre
* Tesseroids e Colaboração com Carla
* AGU 2010
* GOCE 2011
* Fatiando
* Scipy 2013 e 2014
* Boletim sbgf
* Palestra USP, ON e UH
* GMT
* Scipy  2017

Pesquisa
* Posição sobre pesquisa e ciência aberta
* Iniciação na paleo
* Iniciação com tesseroides
* Sementes
* Colaboração com Dio
* Colaboração com Figura.
* Colaborações com Biroca: PEL, magdir
* Tutorial Euler
* Inversão da Moho
* Tutorial NMO
* Posdoc GMT

Ensino
* Posição sobre ensino
* Minicurso na USP
* Minicurso UnB
* Minicurso Python USP
* Minicurso UERJ
* Minicurso UH
* Entrada UERJ
* Disciplinas da UERJ
* Homenagem
* Orientação do Vinícius e Fernanda.
* Qualitec e Victor


