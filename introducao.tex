\section{Introdução}


Tive meu primeiro contato com a ciência e o ensino superior observando o
trabalho de meus pais, ambos Professores do Instituto de Biociências da
Universidade Estadual Paulista ``Júlio de Mesquita Filho'' (UNESP) de Botucatu,
São Paulo.
Até hoje associo as bancadas de microscópios e o cheiro de formol com o
conceito de ciência, mesmo nunca tendo trabalho em um laboratório com essas
características.
A curiosidade, dedicação e ética de meus pais formou a base da minha posição a
respeito da ciência e o que significa ser um educador.

Ingressei no curso de Bacharelado em Geofísica da Universidade de São
Paulo (USP) em 2004.
De agosto de 2008 a maio de 2009 realizei um intercâmbio internacional na York
University, Canadá, onde busquei aprimorar meus conhecimentos de geodésia e
gravimetria.
No final de 2009 concluí o Bacharelado defendendo o trabalho de conclusão de
curso intitulado ``Cálculo do tensor gradiente gravimétrico utilizando
tesseroides'' sob orientação da Profa. Dra. Naomi Ussami
(disponível em \url{https://doi.org/10.6084/m9.figshare.963547}).
Em 2010 iniciei o Mestrado em Geofísica no Observatório Nacional (ON) com bolsa
da CAPES e sob orientação da Profa. Dra. Valéria C. F. Barbosa.
Defendi minha dissertação de Mestrado intitulada ``Robust 3D gravity gradient
inversion by planting anomalous densities'' em outubro de 2011 (disponível em
\url{http://www.leouieda.com/about/masters.html}).
Em novembro de 2011 dei início ao Doutorado em Geofísica no ON, ainda sob
orientação da Profa. Valéria e novamente com bolsa da CAPES.
Em abril de 2016 defendi minha tese de Doutorado intitulada ``Modelagem direta
e inversão de campos gravitacionais em coordenadas esféricas''
(disponível em \url{http://www.leouieda.com/about/phd.html}).

Em outubro de 2013 prestei e fui aprovado no concurso público para o cargo de
Professor Assistente no Departamento de Geologia Aplicada da Faculdade de
Geologia da Universidade do Estado do Rio de Janeiro (UERJ).
Tomei posse na UERJ em fevereiro de 2014, cerca de 2 anos antes de concluir o
Doutorado.
Sou responsável pelo Laboratório de Geofísica de Exploração (LAGEX) e pelas
disciplinas Geofísica 1 e 2 do curso de Bacharelado em Geologia e Matemática
Especial 1 do curso de Bacharelado em Oceanografia.
Em 2016 tive imensa honra e felicidade de ser homenageado como Paraninfo da
turma de formandos do curso de Geologia da UERJ.

Em outubro de 2016 fui selecionado para realizar um pós-doutorado na University
of Hawaii sob supervisão do Prof. Paul Wessel.
O projeto, financiado pela National Science Foundation (NSF), terá duração de 2
anos e tem como objetivo o desenvolvimento de uma interface na linguagem de
programação Python para o software Generic Mapping Tools (GMT), criado e
desenvolvido pelo Prof. Wessel e colaboradores.
Em fevereiro de 2017 recebi licença da UERJ para o pós-doutorado e me mudei
para Honolulu, Havaí, E.U.A., onde me encontro atualmente.

Sou autor de 10 artigos publicados nos periódicos científicos internacionais
{\em Nonlinear Processes in Geophysic},
{\em The Leading Edge},
{\em Journal of Applied Geophysics},
{\em Ore Geology Reviews},
{\em Geophysics} e
{\em Geophysical Journal International}.
Publiquei 7 resumos e 11 trabalhos completos em anais do eventos (2 nacionais e
16 internacionais).
Fui o apresentador de 12 desses trabalhos nos eventos
{\em EAGE Conference and Exhibition},
{\em SEG International Exposition and Annual Meeting},
{\em International Congress of the Brazilian Geophysical Society},
{\em International GOCE User Workshop},
{\em International Symposium on Gravity, Geoid and Height Systems},
{\em AGU Meeting of the Americas},
{\em EGU General Assembly} e
{\em Python in Science Conference (Scipy)}.
Sou o criador e principal desenvolvedor dos programas de código aberto
(software livre)
{\em Tesseroids} (\url{http://tesseroids.leouieda.com}),
{\em Fatiando a Terra} (\url{http://www.fatiando.org}) e
{\em GMT/Python} (\url{https://github.com/GenericMappingTools/gmt-python}).


Desde meu primeiro contato com a programação fui encantado e influenciado pelos
ideais do movimento software livre ({\em free software movement}),
principalmente pela transparência, colaboração e reutilização que se encontra
nessa comunidade.
Esses ideais também são a base da chamada ``Ciência Aberta'' e se transferiram
naturalmente para minha carreira de pesquisa e ensino.
Disponibilizo na internet o material didático de minhas disciplinas, as
apresentações que elaboro e o código fonte utilizado em minhas publicações como
primeiro autor.
Faço essa distribuição principalmente através das páginas
{\em Github} (\url{https://github.com/leouieda}) e
{\em figshare} (\url{http://figshare.com/authors/Leonardo%20Uieda/97471})
e de minha página pessoal (\url{http://www.leouieda.com}),
que agrega esse material.
Como um esforço para aumentar a transparência do meu trabalho, tenho publicado
em minha página textos sobre o andamento de meus projetos.

Além dos artigos referentes a tese, 6
são fruto de colaborações e desses 4 tem como base o Fatiando a Terra.
Os produtos de maior impacto resultantes de minha tese são os programas
Tesseroids e Fatiando a Terra. Um fator importante para seu sucesso foi seu
desenvolvimento no formato de código aberto desde sua concepção. As citações de
ambos programas em artigos científicos demonstram que ambos são utilizados
internacionalmente. O Fatiando a Terra recebeu contribuições de 13 pessoas de 6
países e tem aplicações na pesquisa, no ensino e na divulgação da geofísica.
Utilizei o software para ministrar três minicursos e como parte integral das
disciplinas de geofísica que ministro na UERJ. Atualmente, o Fatiando é
utilizado na maioria dos trabalhos do grupo de pesquisa de problemas inversos
do ON.







Ciência aberta e código no github e site (coisas na net).

Papers: quantos, colaborações.

Trabalhos em congressos: quantos, colaborações, quais congressos.

Como vou organizar esse memorial.
Divido em três temas.

O primeiro, Software, está por trás de todo meu trabalho nos demais.
Desenvolvo esses programas para facilitar minha pesquisa, complementar e
fortalecer minhas técnicas de ensino e como forma de conectar com a comunidade
da geofísica nacional e internacional.

Tive dificuldade em elaborar uma organização coerente para esses temas.
O desenvolvimento de software esteve presente ao longo de toda minha carreira.
Uma ordem cronológica seria muito confuso por que os temas se misturam contribuem entre si.

Software

Pesquisa

Ensino

\citet{uieda2016a}

