\documentclass[12pt,a4paper,oneside,titlepage,onecolumn]{article}


% Para Português
\usepackage[utf8]{inputenc}
\usepackage[brazil]{babel}

% Insert pages from PDFs
\usepackage{pdfpages}
% Generate filler text
\usepackage{lipsum}
% Control the font size
\usepackage{anyfontsize}
% For inserting figures
\usepackage{graphicx}
% Mathematical Symbols and fonts
\usepackage{amssymb,amsmath}
% Bibliography citations
\usepackage[round]{natbib}
% To insert quotations
\usepackage{csquotes}

% Set margins
\usepackage{setspace}
\usepackage[a4paper]{geometry}
\geometry{verbose,tmargin=3cm,bmargin=3cm,lmargin=3cm,rmargin=3cm}

% Metadata and hyperlinks
\usepackage[colorlinks=true]{hyperref}
\newcommand{\Title}{Memorial de Leonardo Uieda}
\hypersetup{
    pdftitle={\Title},
    pdfauthor={Leonardo Uieda},
    pdfsubject={Memorial para concurso de Professor Doutor na USP},
    pdfcreator={pdfTeX},
    allcolors=black,
}

% Make the fancy header
\usepackage{fancyhdr}
\pagestyle{fancy}
\lhead{}
\cfoot{\thepage}


\begin{document}

\begin{titlepage}
    \begin{center}
        %\hspace{\fill}
        %\\[2cm]

        \begingroup
        \fontfamily{bch}\selectfont
        {\large LEONARDO UIEDA}
        \\[9.5cm]

        {\fontsize{40}{0}\selectfont \bfseries MEMORIAL}
        \vfill

        {\large 2017}
        \endgroup
    \end{center}
\end{titlepage}


\begin{titlepage}
    \begin{center}
        \hspace{\fill}
        \\[6cm]

        {\Huge \textbf{Memorial}}
        \\[3cm]

        {\Large Leonardo Uieda}
        \\[3cm]

        \begin{flushright}
        \parbox{9cm}{\large
            Apresentado para concurso público de títulos e provas para cargo de
            Professor Doutor junto ao Departamento de Geofísica do Instituto de
            Astronomia, Geofísica e Ciências Atmosféricas da Universidade de
            São Paulo.
        }
        \end{flushright}
        \vfill

        {\large 2017}
    \end{center}
\end{titlepage}


\pagenumbering{roman}
\singlespacing

\thispagestyle{plain}
\tableofcontents
\newpage

\pagenumbering{arabic}
\onehalfspacing

%%%%%%%%%%%%%%%%%%%%%%%%%%%%%%%%%%%%%%%%%%%%%%%%%%%%%%%%%%%%%%%%%%%%%%%%%%%%%%%%
\section{Introdução}

Tive meu primeiro contato com a ciência e o ensino superior observando o
trabalho dos meus pais, ambos professores do Instituto de Biociências da
Universidade Estadual Paulista ``Júlio de Mesquita Filho'' (UNESP) de Botucatu,
São Paulo.
Até hoje associo as bancadas de microscópios e o cheiro de formol com o
conceito de ciência, mesmo nunca tendo trabalho em um laboratório com essas
características.
A curiosidade, a dedicação e a ética dos meus pais formaram a base da minha
posição a respeito da ciência e do que significa ser um educador.

Ingressei no curso de Bacharelado em Geofísica da Universidade de São
Paulo (USP) em 2004.
De agosto de 2008 a maio de 2009 realizei um intercâmbio internacional na York
University, Canadá, onde busquei aprimorar meus conhecimentos de geodésia e de
gravimetria.
No final de 2009 concluí o Bacharelado defendendo o trabalho de conclusão de
curso intitulado
``Cálculo do tensor gradiente gravimétrico utilizando tesseroides''
(disponível em \url{https://doi.org/10.6084/m9.figshare.963547})
sob orientação da Profa. Dra. Naomi Ussami e em colaboração com a
Profa. Dra. Carla Braitenberg da University of Trieste, Itália.
Em 2010 iniciei o Mestrado em Geofísica no Observatório Nacional (ON) com bolsa
da CAPES e sob orientação da Profa. Dra. Valéria C. F. Barbosa.
A convite da Profa. Braitenberg, passei o mês de fevereiro de 2011 em Trieste
dando continuidade ao projeto que havia começado durante a graduação
(o software {\em Tesseroids}).
Defendi minha dissertação de Mestrado intitulada ``Robust 3D gravity gradient
inversion by planting anomalous densities'' em outubro de 2011 (disponível em
\url{http://www.leouieda.com/about/masters.html}).
Em novembro de 2011 dei início ao Doutorado em Geofísica no ON, ainda sob
orientação da Profa. Valéria e novamente com bolsa da CAPES.
Em abril de 2016 defendi minha tese de Doutorado intitulada ``Modelagem direta
e inversão de campos gravitacionais em coordenadas esféricas''
(disponível em \url{http://www.leouieda.com/about/phd.html}).

Em outubro de 2013 prestei e fui aprovado no concurso público para o cargo de
Professor Assistente no Departamento de Geologia Aplicada da Faculdade de
Geologia da Universidade do Estado do Rio de Janeiro (UERJ).
Tomei posse na UERJ em fevereiro de 2014, cerca de 2 anos antes de concluir o
Doutorado.
Sou responsável pelo Laboratório de Geofísica de Exploração (LAGEX) e pelas
disciplinas Geofísica 1 e 2 do curso de Bacharelado em Geologia e Matemática
Especial 1 do curso de Bacharelado em Oceanografia.
Em 2016 tive a imensa honra e felicidade de ser homenageado como Paraninfo da
turma de formandos do curso de Geologia da UERJ.

Em outubro de 2016 fui selecionado para realizar um pós-doutorado na University
of Hawaii, E.U.A., sob supervisão do Prof. Dr. Paul Wessel.
O projeto terá duração de 2 anos e é financiado pela National Science
Foundation (NSF).
Seu objetivo é o desenvolvimento de uma interface na linguagem de programação
Python para o software Generic Mapping Tools (GMT), criado e desenvolvido pelo
Prof. Wessel e colaboradores.
Em fevereiro de 2017 recebi licença da UERJ para realizar o pós-doutorado e me
mudei para Honolulu, Havaí, E.U.A., onde me encontro atualmente.

Sou autor de 10 artigos publicados nos periódicos científicos internacionais
{\em Nonlinear Processes in Geophysic},
{\em The Leading Edge},
{\em Journal of Applied Geophysics},
{\em Ore Geology Reviews},
{\em Geophysics} e
{\em Geophysical Journal International}.
Publiquei 7 resumos e 11 trabalhos completos em anais de eventos (2 nacionais e
16 internacionais).
Fui o apresentador de 12 desses trabalhos nos eventos
{\em EAGE Conference and Exhibition},
{\em SEG International Exposition and Annual Meeting},
{\em International Congress of the Brazilian Geophysical Society},
{\em International GOCE User Workshop},
{\em International Symposium on Gravity, Geoid and Height Systems},
{\em AGU Meeting of the Americas},
{\em EGU General Assembly} e
{\em Python in Science Conference (Scipy)}.
Atuo como revisor dos periódicos
{\em Computers \& Geosciences},
{\em Geophysics},
{\em Central European Journal of Geosciences (Open Geosciences)},
{\em Pure and Applied Geophysics},
{\em Journal of Applied Geophysics},
{\em Geophysical Prospecting}
e {\em Geophysical Journal International}.
Sou o criador e o principal desenvolvedor dos programas de código aberto
(software livre)
{\em Tesseroids} (\url{http://tesseroids.leouieda.com}),
{\em Fatiando a Terra} (\url{http://www.fatiando.org}) e
{\em GMT/Python} (\url{https://github.com/GenericMappingTools/gmt-python}).

Desde meu primeiro contato com a programação fui encantado e influenciado pelos
ideais do movimento software livre ({\em free software movement}),
principalmente pela transparência, colaboração e reutilização que acontece
nessa comunidade.
Esses ideais também foram adotados pela chamada ``Ciência Aberta'' e se
transferiram naturalmente para minha carreira de pesquisa e de ensino.
Disponibilizo na internet o material didático de minhas disciplinas, as
apresentações que elaboro e o código fonte utilizado em minhas publicações como
primeiro autor.
Faço essa distribuição principalmente através das páginas
{\em Github} (\url{https://github.com/leouieda} e
\url{https://github.com/pinga-lab}) e
{\em figshare} (\url{http://figshare.com/authors/Leonardo%20Uieda/97471})
e de minha página pessoal (\url{http://www.leouieda.com}).
Como um esforço para aumentar a transparência do meu trabalho, tenho publicado
em minha página textos sobre o andamento dos meus projetos.

Sem dúvida, os produtos de maior impacto resultantes de minha pesquisa são os
programas {\em Tesseroids} e {\em Fatiando a Terra}.
As citações em artigos científicos que ambos os programas
receberam\footnote{Segundo a base Google Scholar
(\url{https://scholar.google.com.br/citations?user=qfmPrUEAAAAJ&hl=en}).}
demonstram que são utilizados internacionalmente.
Um fator importante para seu sucesso foi seu desenvolvimento no formato de
software livre desde sua concepção.
Além disso, tenho me esforçado para recrutar novos desenvolvedores e
colaboradores para os projetos.
Atualmente, o {\em Fatiando a Terra} recebeu contribuições de 13 pessoas de 6
países
diferentes\footnote{\url{http://www.fatiando.org/dev/contributors.html}},
7 das quais eu ainda não conheci em pessoa.
Utilizo o {\em Fatiando} como parte integral de minha pesquisa e das
disciplinas e cursos que ministro.
Atualmente, esse programa é utilizado como a base da maioria dos trabalhos
desenvolvidos pelo Grupo de Pesquisa em Problemas Inversos em Geofísica
(\url{http://www.pinga-lab.org}), do qual faço parte.
\\[0.5cm]

A seguir, apresento uma análise reflexiva sobre os principais temas de minha
vida acadêmica: meus projetos de software livre, minha pesquisa em problemas
inversos e minhas atividades de ensino.

%%%%%%%%%%%%%%%%%%%%%%%%%%%%%%%%%%%%%%%%%%%%%%%%%%%%%%%%%%%%%%%%%%%%%%%%%%%%%%%%
\section{Software}
\label{software}

Cada vez mais, a pesquisa de ponta está envolvida com o desenvolvimento de
algum tipo de software.
Isso é verdade para diversos ramos da ciência, da biologia à geofísica e até
mesmo nas ciências sociais.
Aplicações variam desde baixar da internet para um computador pessoal os dados
de fontes governamentais até realizar tomografia sísmica de ponta em um
\textit{cluster}.
Na ciência moderna, muitas vezes o software assume o papel da metodologia
experimental.
Logo é de se esperar que os programas e seus códigos fonte sejam submetidos
ao mesmo rigor que impomos ao resto da ciência experimental.
Infelizmente essa não é a realidade ainda.
Nos próximos 10 anos, questões envolvendo a avaliação da qualidade e da
confiabilidade de resultados computacionais serão centrais no cenário
científico internacional.
No entanto, o sistema de educação superior atual fornece pouco treinamento nas
habilidades necessárias para lidar com essa nova realidade.
Algumas iniciativas, como o \textit{Software Carpentry}
(\url{https://software-carpentry.org/}), estão atacando diretamente esses
problemas oferecendo treinamento computacional direcionado a cientistas.

Minha pesquisa sempre teve um viés computacional.
Ao longo dos últimos dez anos, dei início a três projetos de software livre.
A seguir, discorro sobre minha formação em desenvolvimento de software e sobre
a concepção e os impactos de cada projeto.


\subsection{Formação}

Meu primeiro contato com a programação foi através da disciplina
``Introdução à Computação para Ciências Exatas e Tecnologia''
que cursei em 2004 durante meu primeiro semestre na USP.
Antes disso, eu não tinha conhecimento algum de como programas de computador
são feitos ou que qualquer um poderia criar o seu próprio programa.
Aprendi os conceitos básicos da linguagem de programação C.
Porém, não acalcei um nível suficientemente avançado para enxergar aplicações
imediatas da programação nas demais disciplinas do curso de geofísica.

Busquei aprender mais sobre a linguagem C através da disciplina optativa
``Computação para Geofísicos''.
Durante a disciplina, desenvolvi aplicações diretas da programação à geofísica
como o cálculo do International Geomagnetic Reference Field (IGRF) a partir dos
coeficientes de harmônicos esféricos e o método de \citet{talwani1959} para
modelagem direta na gravimetria.
O código que criei para a disciplina foi utilizado anos depois como base
para a implementação do método de \citet{talwani1959} no programa
\textit{Fatiando a
Terra}\footnote{\url{www.fatiando.org/v0.5/api/gravmag.talwani.html}}.
Essas aplicações me mostraram o enorme poder da programação no aprendizado de
conceitos complexos da geofísica e da matemática.
Ao criar uma implementação computacional de um método, o aluno é levado a
considerar detalhes e a elaborar perguntas que passariam despercebidas ao
estudar somente pela teoria.
Além disso, também é capaz de explorar as possibilidades e os limites de uma
teoria de forma dinâmica e independente.
Não exagero quando afirmo que ter cursado a disciplina ``Computação para
Geofísicos'' foi crucial para o resto de minha carreira.

Nos anos seguintes comecei a estudar a programação nas horas vagas e a aplicar
à geofísica o que estava aprendendo.
Implementei a Transformada Discreta de Fourier\footnote{Disponível em
\url{https://github.com/leouieda/dft-in-c}} para estudar para a disciplina
``Processamento de Sinais Digitais''.
Utilizei minha implementação do método de otimização Ant Colony Optimization
\citep{socha2008} para realizar uma inversão das velocidades de grupo de ondas
Love\footnote{Disponível em \url{https://github.com/leouieda/love-aco-inv}}
para a disciplina ``Teoria de Ondas Sísmicas e Estrutura da Terra''.
Para as disciplinas de geodésia que cursei durante meu intercâmbio na York
University, implementei ajustes de redes gravimétricas, mudança de sistemas de
coordenadas geográficas, entre outros.
Aprendi as linguagens de programação C++, Java e Python.
Cursei a disciplina optativa ``Princípios de Desenvolvimento de Algoritmos''
onde aprendi os conceitos que possibilitaram alguns dos avanços que obtive em
\citet{tesseroids}.

Em 2008 descobri a organização sem fins lucrativos \textit{Software Carpentry}.
Segundo \url{https://software-carpentry.org} (acessado em 7 de julho de 2017),
seu objetivo é ``Ensinar habilidades laboratoriais básicas para pesquisa
computacional''.
Através de seu material, disponível gratuitamente na internet, descobri o
quanto eu não sabia sobre o desenvolvimento sustentável de software para
pesquisa científica.
Aprendi sobre ferramentas fundamentais da computação que até então eu
desconhecia, como sistemas de controle de versão, testes unitários e
rastreamento da procedência de resultados.
Atualmente utilizo essas ferramentas no meu dia-a-dia como pesquisador e
professor, seja para escrever
artigos\footnote{\url{https://github.com/pinga-lab/paper-moho-inversion-tesseroids}}
ou para manejar a entrega de trabalhos práticos em minha disciplina de
computação e cálculo numérico\footnote{\url{https://github.com/mat-esp-2016}}.

Durante a pós-graduação, continuei com a abordagem de criar uma implementação
computacional do que aprendia nas disciplinas.
No entanto, optei por agrupar todo código fonte que desenvolvi em uma única
biblioteca feita na linguagem Python: o projeto \textit{Fatiando a Terra}.
Para a disciplina de ondas sísmicas, por exemplo, implementei uma solução por
diferenças finitas da equação da
onda\footnote{\url{www.fatiando.org/v0.5/api/seismic.wavefd.html}} para
investigar se ondas Love realmente resultam de ondas S
horizontais\footnote{\url{https://youtu.be/YjhSvEpbzps}} e se ondas Rayleigh
possuem movimento elíptico
retrógrado\footnote{\url{https://youtu.be/Mvd8FANLqy4}}.
Concomitantemente, continuei aprimorando minhas habilidades com o
desenvolvimento de software através da experiência pessoal com o
\textit{Fatiando a Terra}.



\subsection{Tesseroids}

Em 2007 iniciei um estágio de iniciação científica com a Profa. Dra. Naomi
Ussami.
Meu projeto era parte de uma colaboração com a Profa. Dra. Carla Braitenberg da
University of Trieste, Itália.
O objetivo dessa colaboração era preparar a comunidade científica para lidar
com os dados de gradiometria gravimétrica que seriam coletados pelo satélite
GOCE (Gravity field and steady-state Ocean Circulation Explorer).
Minha participação nesse projeto seria desenvolver um programa para a modelagem
direta dos dados utilizando tesseroides (prismas esféricos).
Ficou determinado que o método que eu utilizaria para isso seria a Quadratura
Gauss-Legendre, como proposto por \citet{asgharzadeh2007} e
\citet{wild-pfeiffer2008}.
Decidi realizar minha implementação dessa metodologia na linguagem Python, a
qual estava aprendendo na época.
No final de 2009 defendi meu trabalho de conclusão de curso e lancei a versão
0.3 do software batizado de \textit{Tesseroids}\footnote{Código fonte da versão
0.3 disponível em \url{https://doi.org/10.5281/zenodo.15804}}.

O programa começou a ser utilizado pelo grupo de pesquisa da Profa.
Braitenberg.
Durante os testes iniciais, descobriram que o tempo de execução do programa era
muito elevado para calcular os efeitos de modelos digitais de terreno
realistas.
Atribuo essa deficiência à minha inexperiência com a linguagem Python, não a
uma limitação da mesma.
A convite da Profa. Braitenberg, fui passar um mês em Trieste, Itália,
para reimplementar o software em linguagem C e torná-lo mais rápido e preciso.
Durante o mês de fevereiro, reescrevi todo o software em C e adicionei
sistemas automáticos de testes unitários e geração de documentação online.
Também implementei um método para a discretização adaptativa dos modelos,
melhorando a estabilidade dos cálculos.
No final de abril, lancei a versão 1.0 do \textit{Tesseroids}.
Essa versão passou a ser utilizada pelo grupo de Trieste e também por outros
grupos de pesquisa da Europa e da América do Sul.
Os seguintes trabalhos, por exemplo, mencionam uso dessa versão do software:
\citet{alvarez2012}, \citet{mariani2013} e \citet{bouman2013}.

Continuei o desenvolvimento do software e da metodologia de discretização
adaptativa durante meu doutorado.
Finalmente, lancei a versão 1.2 em 2015 com alguns avanços metodológicos.
Esta versão acompanha a publicação \citet{tesseroids} na seção \textit{Software
and Alrogithms} da revista \textit{Geophysics}.
Até o momento, meus trabalhos relacionados ao software \textit{Tesseroids}
receberam juntos cerca de 50 citações\footnote{Segundo a base Google Scholar:
\url{https://scholar.google.com.br/citations?user=qfmPrUEAAAAJ&hl=en}}.

Ao olhar todo o histórico do desenvolvimento desse
projeto\footnote{Disponível em
\url{https://github.com/leouieda/tesseroids/commits}},
percebo o quanto aprendi em cada etapa.
Sinto orgulho e felicidade ao perceber que algo que eu criei foi útil para
tantas outras pessoas.
Esse sentimento é o que me motivou a continuar desenvolvendo essas ferramentas
e as disponibilizando de maneira livre e gratuita.



\subsection{Fatiando a Terra}


O \textit{Fatiando a Terra} é uma biblioteca (coleção de funções e classes)
feita na linguagem Python.
Sua funcionalidade abrange diversas áreas da geofísica, com foco em métodos
potenciais.
Comecei seu desenvolvimento durante meu mestrado no Observatório Nacional.
No início, inclui as funções que implementei durante as disciplinas nessa
biblioteca como uma forma de estudo.
Eventualmente, esse código se tornou a base para minha implementação do método
de inversão que desenvolvi como projeto de mestrado \citep[][cujo código está
disponível em
\url{https://github.com/pinga-lab/paper-planting-densities}]{seed}.
Em 2012, fui convidado para ministrar um minicurso de inversão na escola de
verão do IAG/USP junto com meu
amigo Vanderlei C. Oliveira Jr. (na época também aluno da Profa.  Valéria).
Desenvolvemos diversos exercícios práticos para o curso utilizando o
\textit{Fatiando a Terra}.
Como consequência disso, a biblioteca cresceu com a adição de diversas funções
novas.
Também criei a primeira versão da página de documentação
\url{http://www.fatiando.org} e lancei a versão 0.0.1 do programa.
Os exercícios e apresentações criados para esse curso estão disponíveis em
\url{https://github.com/pinga-lab/inversao-iag-2012}.

Trabalhei no desenvolvimento do \textit{Fatiando} continuamente durante todo
meu doutorado.
Em 2013, apresentei sobre o projeto no congresso Scientific Computing with
Python (Scipy) em Austin, E.U.A., resultando no lançamento da versão 0.1 e na
publicação nos anais do congresso:

\begin{displayquote}
    UIEDA, L.; OLIVEIRA JR., V. C.; BARBOSA, V. C. F.  Modeling the Earth with
    Fatiando a Terra. In: 12th Python in Science Conference, 2013, Austin.
    Proceedings of the 12th Python in Science Conference, 2013.
\end{displayquote}

O código fonte utilizado nessa publicação está disponível na página
\url{http://www.leouieda.com/talks/scipy2013.html}. Além disso, há uma gravação
dessa palestra no YouTube (\url{https://www.youtube.com/watch?v=Ec38h1oB8cc}).
Até esse momento, quase todo o desenvolvimento era feito por mim com algumas
colaborações esporádicas de amigos da minha turma de graduação.

Apresentei novamente sobre o \textit{Fatiando} na edição de 2014 do congresso
Scipy (\url{http://www.leouieda.com/posters/scipy2014.html}), dessa vez sobre
os algoritmos de inversão recentemente implementados na biblioteca.
Após esse evento, concentrei meus esforços em melhorar a documentação e
disponibilidade do \textit{Fatiando} com o objetivo de atrair
contribuidores para o projeto.
Percebi que para o projeto crescer, seria necessário envolver outros
programadores.
Com minha entrada na UERJ, essa necessidade se tornou mais evidente.
A carga horária de aulas combinada com o trabalho da tese de doutorado me
deixava pouquíssimo tempo para a programação.
Mesmo assim, o \textit{Fatiando} foi tema de uma entrevista no Boletim da
Sociedade Brasileira de Geofísica número
89\footnote{\url{https://www.sbgf.org.br/home/images/stories/Arquivos/Boletim_89-2014.pdf}}.

Meus esforços de documentar o processo para a submissão de contribuições e de
facilitar a instalação do programa começaram a dar fruto em 2015.
O projeto recebeu contribuições de duas pessoas com as quais eu não tinha
contato prévio\footnote{\url{http://www.fatiando.org/v0.4/contributors.html}}.
Também surgiram os primeiros trabalhos publicados de fora do meu grupo de
pesquisa que mencionam a utilização do \textit{Fatiando}:
\citet{niccoli2015}, \citet{matthews2016} e \citet{bassett2016}.
Dentre esses, destaco \citet{matthews2016} que utilizou a modelagem direta de
dados de gradiometria gravimétrica com prismas retangulares retos.
Os autores compararam a implementação do \textit{Fatiando} com a do software
comercial \textit{ModelVision} da empresa pbEncom, concluindo que a versão do
\textit{Fatiando} proporcionou resultados compatíveis ou até mesmo com maior
acurácia que a do software comercial.
Atualmente, a próxima versão do \textit{Fatiando} (0.6) contará com
contribuições de 13 pessoas, 5 das quais eu não conheço
pessoalmente\footnote{\url{http://www.fatiando.org/dev/contributors.html}}.

Em 2015, fui convidado para ministrar a palestra ``Fatiando a Terra:
construindo uma base para ensino e pesquisa de
geofísica''\footnote{\url{http://www.leouieda.com/talks/iag-04-2015.html}} nos
seminários do Departamento de Geofísica do IAG/USP.
Em 2016, apresentei a mesma palestra nos seminários do Departamento de
Geofísica do Observatório Nacional.
Em 2017, fui convidado a dar a palestra ``Inverting gravity to map the Moho: A
new method and the open source software that made it
possible''\footnote{\url{http://www.leouieda.com/talks/tgif-2017.html}}
na University of Hawaii.
A palestra abordou o \textit{Fatiando a Terra} e a metodologia e os resultados
de \citet{moho}.

A maioria dos artigos e teses recentes do Grupo de Pesquisa em Problemas
Inversos em Geofísica (PINGA), liderado por mim e os Professores do
Observatório Nacional
Vanderlei C. Oliveira Jr e Valéria C. F. Barbosa,
utiliza o \textit{Fatiando a Terra} de alguma forma.
Diversos desses trabalhos disponibilizam seu código fonte livremente através
dos repositórios na página do grupo \url{https://github.com/pinga-lab}.
Por exemplo,
\citet{magdir}\footnote{\url{https://github.com/pinga-lab/Total-magnetization-of-spherical-bodies}},
\citet{reis2016}\footnote{\url{https://github.com/pinga-lab/magnetization-rock-sample}},
\citet{moho}\footnote{\url{https://github.com/pinga-lab/paper-moho-inversion-tesseroids}}
e
\citet{monogenic2017}\footnote{\url{https://github.com/pinga-lab/paper-monogenic-signal}}.

Ao longo dos 7 anos que passei desenvolvendo o \textit{Fatiando},
aprimorei minhas habilidades de programação em Python e aprendi diversas lições
sobre a administração de um projeto de software livre.
Os maiores desafios não estão relacionados à programação diretamente, mas sim à
criação de uma comunidade ativa e empolgada em torno do projeto.


\subsection{GMT/Python}

Após terminar meu doutorado em abril de 2016, comecei a procurar oportunidades
para fazer um pós-doutorado no exterior.
Havia passado os últimos dois anos me familiarizando com a vida de professor
universitário, ministrando de duas a quatro disciplinas por semestre e
trabalhando na minha tese de doutorado durante os recessos.
Estava exausto e ponderando o futuro e o rumo da minha carreira em pesquisa.
Logo, percebi que estava na hora de buscar outra experiência internacional.
Alguns meses depois, recebi um comunicado através de uma lista de emails que o
Prof. Paul Wessel da University of Hawaii, E.U.A., estava procurando candidatos
com experiência em programação em Python para um pós-doutorado.
A responsabilidade do candidato selecionado seria construir uma interface
para acessar os comandos do Generic Mapping Tools (GMT,
\url{http://gmt.soest.hawaii.edu}) através da linguagem Python.
Além disso, também utilizaria essa interface para realizar pesquisas na área de
tectônica de placas e métodos potenciais.
Depois de conversar a respeito com minha esposa, resolvi aplicar para a
posição e fui selecionado\footnote{Escrevi a respeito de minhas escolhas e o
processo de seleção em
\url{http://www.leouieda.com/blog/hawaii-gmt-postdoc.html}}.
Confesso que a perspectiva de morar por dois anos no Havaí contribuiu
de forma não insignificante para minha decisão.
Em fevereiro de 2017 me mudei para Honolulu para iniciar o pós-doutorado.

O projeto \textit{GMT/Python} é uma biblioteca escrita em Python que se
comunica com o GMT através de um mecanismo chamado \textit{foreign function
interface} (FFI).
Esse mecanismo possibilita que funções escritas em uma linguagem de programação
sejam utilizadas em programas feitos em outra linguagem.
Por exemplo, programas feitos em Python podem utilizar a FFI para executar
funções de uma biblioteca escrita em C, a linguagem na qual é feito o GMT.
Uma vantagem do Python é que seu interpretador oficial é implementado em C,
facilitando a interação entre as duas linguagens.
A maior dificuldade que encontrei até o presente momento é a enorme
complexidade do GMT.
Existem dezenas de módulos, cada um com diversas opções e particularidades,
implementados em milhares de linhas de código.

O primeiro passo desse projeto foi auxiliar o Prof. Wessel no desenvolvimento
de um modo de execução moderna (chamado \textit{modern mode}) para o
GMT\footnote{Documentação na página do GMT
\url{http://gmt.soest.hawaii.edu/projects/gmt/wiki/Modernization}}.
Esse modo simplifica a utilização do GMT e introduz comandos novos para
facilitar a criação de figuras complexas com múltiplos gráficos.
A interação com o Prof. Wessel se mostrou indispensável para a elaboração do
\textit{GMT/Python}.
Essa está sendo uma oportunidade única de aprender com alguém que possui
décadas de experiência com software livre.
Outro benefício é a proximidade com a fronteira da pesquisa em tectônica de
placas, que é a especialidade do Prof. Wessel.

Atualmente, o \textit{GMT/Python} está em processo de desenvolvimento.
Já desenvolvi a base necessária para acessar a biblioteca do GMT
e estou no processo de adaptar cada um dos mais de 90 módulos.
Apresentarei sobre os resultados que obtivemos até o momento no congresso
Scientific Computing with Python (Scipy)\footnote{Escrevi sobre o trabalho que
submetemos e as revisões que recebemos em
\url{http://www.leouieda.com/blog/scipy2017-proposal-gmt.html}} em julho de 2017 em Austin, E.U.A.
Optamos por manter o código fonte e nossos planos para o projeto aberto ao
público.
Todo o desenvolvimento e planejamento acontece através da página
\url{https://github.com/GenericMappingTools/gmt-python}.

Este projeto é extremamente promissor devido à enorme base de usuários do GMT
e a crescente popularidade do Python na ciência.
Além disso, atualmente existem poucas alternativas para geração de mapas de
qualidade em Python.
Por exemplo, usuários do software GPlates \citep{gplates} poderão se beneficiar
do \textit{GMT/Python} pois o mesmo possui uma interface com a linguagem Python
e faz amplo uso do GMT.

%%%%%%%%%%%%%%%%%%%%%%%%%%%%%%%%%%%%%%%%%%%%%%%%%%%%%%%%%%%%%%%%%%%%%%%%%%%%%%%%
\section{Pesquisa}

Minha pesquisa se concentra na área de problemas inversos em métodos
potenciais.
Geralmente, meus trabalhos são avanços metodológicos e são acompanhados por
um código fonte que os implementa.
Como mencionei no capítulo \ref{software}, acredito que o código fonte que
acompanha uma publicação é tão importante quanto a descrição de sua
metodologia.
Muitas vezes é impossível reproduzir os resultados de um trabalho
sem ter acesso ao software que os gerou.
Logo, é crucial que o código esteja disponível para ser revisado pela
comunidade científica.
Para tanto, disponibilizo o código e os dados (à medida do possível) necessários
para reproduzir os resultados de meus trabalhos como primeiro autor.
Cada trabalho é acompanhado de um repositório na página do Grupo de Pesquisa em
Problemas Inversos em Geofísica (\url{https://github.com/pinga-lab}).
Os repositórios contém o código fonte que implementa a metodologia, realiza os
testes com dados sintéticos e produz as figuras para o artigo.
Ultimamente, torno público um repositório no momento da submissão do respectivo
artigo para publicação.
Dessa forma, os revisores tem acesso ao conteúdo total de meus trabalhos e
podem se certificarem de que meus resultados estão corretos.
Cada repositório também é arquivado permanentemente e recebe um Digital Object
Identifier (DOI) através de serviços como Zenodo (\url{http://zenodo.org}) e
figshare (\url{https://figshare.com}).

A seguir, apresento reflexões sobre os aspectos de minha formação que me
levaram a essa área de pesquisa e sobre os diferentes trabalhos que formam
minha produção acadêmica.


\subsection{Formação}
%%%%%%%%%%%%%%%%%%%%%%%%%%%%%%%%%%%%%%%%%%%%%%%%%%%%%%%%%%%%%%%%%%%%%%%%%%%%%%%

Desde o início da graduação me senti intrigado pelos métodos de inversão.
Sempre ouvia de alunos veteranos, ou até mesmo de professores, que esse era um
assunto extremamente complexo.
Conhecendo minha personalidade, creio que meu interesse inicial sobre o assunto
era puramente devido ao desafio.
Por sorte, a disciplina que forma a base do conhecimento de problemas inversos,
a álgebra linear, também foi um tema que despertou meu interesse.
Por outro lado, devo confessar que, a princípio, os métodos potenciais não me
interessaram tanto quanto a inversão.
Minha primeira impressão da gravimetria e da magnetometria foi que eram métodos
simples.
Porém, terminei minha primeira disciplina sobre o assunto completamente confuso
e com mais dúvidas do que tinha antes de cursá-la.
Hoje percebo que essas são características de um assunto complexo e
interessante e de uma aula de qualidade.

Fiz minha primeira iniciação científica no laboratório de paleomagnetismo com
bolsa da FAPESP e orientação do Prof. Manoel S. D'Agrella Filho.
Ao final do período de um ano da bolsa, decidi não continuar nessa área.
Em seguida, busquei outro projeto que unisse a geofísica
com meu interesse pela programação.
Em 2007, iniciei o projeto sob orientação da
Profa. Naomi Ussami que resultou no software \textit{Tesseroids}.

Em 2008, realizei um programa de intercâmbio de 10 meses na York University,
Canadá.
Parte do motivo para essa escolha é a excelência do país nas áreas de
gravimetria e geodésia.
Lá, cursei disciplinas sobre geodésia física, gravimetria e ajuste de redes
através do método dos mínimos quadrados.

Em 2010, ingressei no Mestrado em Geofísica do Observatório Nacional sob
orientação da Profa. Valéria C. F. Barbosa.
Meu projeto era criar um método de inversão 3D de dados de gradiometria
gravimétrica, tema que contava com atenção internacional e era pioneiro no
cenário nacional.
Nesse ponto, percebi que as disciplinas cursadas na York me forneceram a base
necessária para compreender a inversão e a gravimetria com muito mais
facilidade.
Ainda sob orientação da Profa. Valéria, ingressei no Doutorado em Geofísica do
Observatório Nacional em novembro de 2011 e defendi minha tese em abril de
2016.
Aprendi com a Profa. Valéria
as diferentes vertentes e sutilezas da inversão,
a arte da elaboração de artigo científico
e sua ética profissional impecável.
Também devo muito do meu aprendizado durante a pós-graduação às longas
conversas e debates com meu amigo Vanderlei C. Oliveira Jr. (atualmente
pesquisador do Observatório Nacional).



\subsection{Modelagem direta de campos gravitacionais com tesseroides}
%%%%%%%%%%%%%%%%%%%%%%%%%%%%%%%%%%%%%%%%%%%%%%%%%%%%%%%%%%%%%%%%%%%%%%%%%%%%%%%

Comecei a desenvolver esse tema durante minha iniciação científica com a
Profa. Naomi de 2007 a 2009.
A princípio, meu trabalho era simplesmente aprender a metodologia de
\citet{wild-pfeiffer2008} e transformá-la em um programa de computador.
Este foi o primeiro trabalho que apresentei em evento internacional:

\begin{displayquote}
    UIEDA, L.; USSAMI, N.; BRAITENBERG, C. Computation of the gravity
    gradient tensor due to topographic masses using tesseroids. In: AGU Meeting
    of the Americas, 2010.\footnote{Apresentação disponível em
    \url{http://www.leouieda.com/talks/agu2010.html}}
\end{displayquote}

Retomei o projeto em 2011 durante minha estadia em Trieste com a Profa.
Braitenberg para atualizar o software e a metodologia para o cálculo dos campos
gravitacionais de um tesseroide (prisma esférico).
Utilizei as equações otimizadas de \citet{grombein2013} para eliminar
as singularidades presentes na formulação de \citet{wild-pfeiffer2008}.
Também desenvolvi um método de discretização adaptativa dos tesseroides para
garantir a acurácia da integração numérica com a Quadratura Gauss-Legendre.
Sem meu conhecimento, um algoritmo similar \citep{li2011} havia sido publicado
no mesmo período em que estava em Trieste, inviabilizando a nossa publicação do
método.
Ainda em Trieste, busquei caracterizar o erro numérico envolvido
nos cálculos para melhor controlar a discretização adaptativa.
Publiquei as atualizações do software \textit{Tesseroids} e os resultados da
análise do erro da quadratura nos anais do 4th International GOCE User
Workshop:

\begin{displayquote}
    UIEDA, L.; BOMFIM, E. P.; BRAITENBERG, C.; MOLINA, E. C. Optimal
    forward calculation method of the Marussi tensor due to a geologic
    structure at GOCE height. In: 4th International GOCE User Workshop,
    2011.\footnote{Pôster e texto disponíveis em
    \url{http://www.leouieda.com/posters/goce2011.html}}
\end{displayquote}

Para minha tese de doutorado iria utilizar a modelagem direta com tesseroides
para desenvolver um método de inversão em coordenadas esféricas.
Porém, para que a inversão seja correta é necessário que a modelagem direta
seja a mais precisa possível.
Logo, continuei com o desenvolvimento do algoritmo de discretização adaptativa
e com os experimentos para caracterizar o erro da quadratura.
Após diversas tentativas frustadas, cheguei aos resultados apresentados no
artigo que fez parte da minha tese de doutorado:

\begin{displayquote}
    UIEDA, L; BARBOSA, V. C. F.; BRAITENBERG, C. Tesseroids:
    Forward-modeling gravitational fields in spherical coordinates. Geophysics,
    v. 81, p. F41-F48, 2016.\footnote{Código fonte disponível em
    \url{https://github.com/pinga-lab/paper-tesseroids}}
\end{displayquote}

Neste trabalho, aprimoramos o algoritmo de discretização adaptativa proposto
por \citet{li2011}.
Também determinamos empiricamente valores para o parâmetro
\textit{distance-size ratio}, que controla a discretização adaptativa,
para manter o erro de integração abaixo de $0.1\%$.

Os trabalhos relacionados à modelagem direta com tesseroides estão entre os
meus trabalhos mais citados\footnote{Segundo a base Google Scholar:
\url{https://scholar.google.com.br/citations?user=qfmPrUEAAAAJ&hl=en}}.
Acredito que isso seja devido, em parte, à disponibilização do software
\textit{Tesseroids} que implementa essa metodologia.
Outro fruto dessa pesquisa é minha coorientação da tese de doutorado do aluno
Santiago Soler com o Prof. Dr. Mario Ernesto Gimenez da Universidad Nacional de
San Juan, Argentina.
A tese de Santiago dará continuidade à modelagem direta com tesseroides,
expandindo a formulação para incluir tesseroides com distribuições internas de
densidades variáveis.


\subsection{Inversão 3D utilizando o método de plantação}
%%%%%%%%%%%%%%%%%%%%%%%%%%%%%%%%%%%%%%%%%%%%%%%%%%%%%%%%%%%%%%%%%%%%%%%%%%%%%%%

Meu projeto de mestrado era desenvolver um método de inversão 3D para dados de
gradiometria gravimétrica.
Um dos desafios enfrentados nessa área é o aumento significativo do número de
dados em uma aquisição gradiométrica comparados com uma aquisição
aerogravimétrica.
Esse aumento tornava a inversão impossível de ser executada nos computadores
que tínhamos disponíveis no Observatório Nacional.

Minha ideia inicial para o método de inversão surgiu durante uma conversa com o
Prof. Dr. João B. C. Silva da Universidade Federal do Pará.
No momento, ele se encontrava no Rio de Janeiro para participar de uma banca de
mestrado.
Durante a conversa, o Prof. João mencionou o trabalho de \citet{rene} como um
exemplo de uma metodologia diferente e pouco reconhecida pela comunidade
científica.
\citet{rene} propôs um método de inversão 2D de dados gravimétricos na qual a
solução cresce em torno de alguns elementos nucleares chamados de ``sementes''.
O que despertou meu interesse nesse trabalho foi o viés computacional do método
proposto, ao invés da abordagem matemática clássica.
Ao estudá-lo, percebi que poderia ser adaptado e aprimorado para o
caso 3D e para dados de gradiometria gravimétrica.
Minhas modificações incluem a introdução da função de regularização de
compacidade de \citet{silvadias2009},
um termo de normalização para as funções do ajuste
e a avaliação parcial da matriz de sensibilidade.
Esta última modificação é a que possibilita a inversão de conjuntos grandes de
dados com poucos recursos computacionais.

Apresentei este novo método, denominado ``método de plantação'',  nos
congressos internacionais
73rd EAGE Conference and Exhibition incorporating SPE EUROPEC,
SEG International Exposition and Eighty-First Annual Meeting,
12th International Congress of the Brazilian Geophysical Society,
e
International Symposium on Gravity, Geoid and Height Systems.
Fui premiado com o auxílio financeiro da Near Surface Geophysics Section (NSGS)
da SEG para participar do SEG Annual Meeting.
Também obtive o auxílio PACE Student Travel Grant para participar do 73rd EAGE
Conference and Exhibition.

Publiquei trabalhos completos nos anais de 3 desses eventos:

\begin{displayquote}
    UIEDA, L.; BARBOSA, V. C. F. 3D gravity Gradient Inversion by Planting
    Density Anomalies. In: 73rd EAGE Conference and Exhibition incorporating
    SPE EUROPEC, 2011, Vienna. v. 1.\footnote{Pôster e código fonte:
    \url{http://www.leouieda.com/posters/eage2011.html}}
\end{displayquote}

\begin{displayquote}
    UIEDA, L.; BARBOSA, V. C. F. Robust 3D gravity gradient
    inversion by planting anomalous densities. In: SEG International Exposition
    and Eighty-First Annual Meeting, 2011, San Antonio. p.
    820-824.\footnote{Apresentação e código fonte:
    \url{http://www.leouieda.com/talks/seg2011.html}}
\end{displayquote}

\begin{displayquote}
    UIEDA, L.; BARBOSA, V. C. F. 3D gravity inversion by planting
    anomalous densities. In: 12th International Congress of the Brazilian
    Geophysical Society, Rio de Janeiro, Brazil,
    2011.\footnote{Apresentação e código fonte:
    \url{http://www.leouieda.com/talks/sbgf2011.html}}
\end{displayquote}

Subsequentemente, publiquei meu primeiro artigo em periódico:

\begin{displayquote}
    UIEDA, L.; BARBOSA, V. C. F. Robust 3D gravity gradient inversion by
    planting anomalous densities. Geophysics, v. 77, p. G55-G66,
    2012.\footnote{Código fonte disponível em
    \url{https://github.com/pinga-lab/paper-planting-densities}}
\end{displayquote}

O próximo passo no desenvolvimento desse método veio quando, ao ler novamente o
trabalho original de \citet{rene}, percebi as enormes vantagens da utilização
da função ``shape-of-anomaly''.
Essa função é definida e utilizada no trabalho de 1986.
Porém, acredito que René não explorou todas suas implicações para a inversão.
Ao utilizá-la em meu método de plantação, percebi que era capaz de obter
resultados melhores utilizando menos elementos nucleares (as ``sementes'' da
inversão).
Apresentei estes resultados e uma análise do motivo de seu sucesso no congresso
SEG International Exposition and Eighty-Second Annual Meeting,
acompanhado da publicação nos anais do evento:

\begin{displayquote}
    UIEDA, L.; BARBOSA, V. C. F. Use of the shape-of-anomaly data
    misfit in 3D inversion by planting anomalous densities. In: SEG Technical
    Program Expanded Abstracts 2012.\footnote{Apresentação e código fonte:
    \url{http://www.leouieda.com/talks/seg2012.html}}
\end{displayquote}


Durante meu doutorado, adaptei este método para a inversão de dados de anomalia
magnética de campo total com resultados insatisfatórios.
Os resultados da inversão se mostraram extremamente sensíveis à direção de
magnetização total assumida para o alvo.
Não dei continuidade com esse trabalho pois o foco de minha tese havia mudado
para a inversões em escala regional.
Apresentei meus resultados no evento AGU Meeting of the Americas de 2013:

\begin{displayquote}
    UIEDA, L.; BARBOSA, V. C. F. 3D magnetic inversion by planting anomalous
    densities. In: AGU Meeting of the Americas, 2013,
    Cancun.\footnote{Apresentação e código fonte:
    \url{http://www.leouieda.com/talks/agu-cancun2013.html}}
\end{displayquote}

Em seguida, adaptei o método de plantação para utilizar tesseroides ao invés de
prismas retangulares retos.
Dessa forma, poderia realizar a inversão em coordenadas esféricas e em escala
regional.
No entanto, o método de plantação assume que os alvos da inversão são corpos
contínuos e com contraste de densidade abrupto em relação às estruturas
encaixantes.
Embora o método funcione em testes com dados sintéticos,
tive dificuldade de encontrar situações reais onde o método se aplicaria.
Apresentei esses resultados no EGU General Assembly de 2014:

\begin{displayquote}
    UIEDA, L.; BARBOSA, V. C. F. Gravity inversion in spherical
    coordinates using tesseroids. In: EGU General Assembly 2014.
    EGU2014-10898-1.\footnote{Apresentação e código fonte:
    \url{http://www.leouieda.com/talks/egu2014.html}}
\end{displayquote}


O método de plantação foi utilizado na tese de doutorado de Dionísio U. Carlos,
que na época era aluno da Profa. Valéria, para modelar dados de gradiometria do
Quadrilátero Ferrífero.
Dessa colaboração com o Dionísio foram publicados 3 trabalhos completos em
anais de eventos e 2 artigos em periódicos internacionais:

\begin{displayquote}
    CARLOS, D. U.; BARBOSA, V. C. F.; UIEDA, L.; BRAGA, M. A. Inversão de
    dados de aerogradiometria gravimétrica 3D-FTG aplicada a exploração mineral
    na região do Quadrilátero Ferrífero. In: 12th International Congress of the
    Brazilian Geophysical Society, 2011.
\end{displayquote}

\begin{displayquote}
    CARLOS, D. U.; UIEDA, L.; BARBOSA, V. C. F.; BRAGA,
    M. A.; GOMES, A. A. S. In-depth imaging of an iron
    orebody from Quadrilatero Ferrifero using 3D gravity gradient inversion.
    In: SEG International Exposition and Eighty-First Annual Meeting, 2011.
\end{displayquote}

\begin{displayquote}
    CARLOS, D. U,; UIEDA, L.; LI, Y.; BARBOSA, V.
    C. F.; BRAGA, M. A. ; ANGELI, G.; PERES, G. Iron ore
    interpretation using gravity-gradient inversions in the Carajás, Brazil.
    In: SEG Technical Program Expanded Abstracts 2012.
\end{displayquote}

\begin{displayquote}
    CARLOS, D. U.; UIEDA, L.; BARBOSA, V. C. F. Imaging
    iron ore from the Quadrilátero Ferrífero (Brazil) using geophysical
    inversion and drill hole data. Ore Geology Reviews, v. 61, p. 268-285,
    2014.
\end{displayquote}

\begin{displayquote}
    CARLOS, D. U.; UIEDA, L.; BARBOSA, V. C. F. How two
    gravity-gradient inversion methods can be used to reveal different geologic
    features of ore deposit - A case study from the Quadrilátero Ferrífero
    (Brazil). Journal of Applied Geophysics, v. 130, p. 153-168, 2016.
\end{displayquote}


O método de plantação \citep{seed} é meu trabalho mais citado\footnote{Segundo
a base Google Scholar:
\url{https://scholar.google.com.br/citations?user=qfmPrUEAAAAJ&hl=en}} e o que
me rendeu o maior número de publicações.
Este trabalho marcou minha iniciação
à apresentação de trabalhos em congressos no exterior e
à escrita e publicação de artigos científicos.
Também serviu como uma primeira tentativa de tornar a minha pesquisa mais
transparente, acessível e reprodutível.


\subsection{Inversão 3D em coordenadas esféricas}
%%%%%%%%%%%%%%%%%%%%%%%%%%%%%%%%%%%%%%%%%%%%%%%%%%%%%%%%%%%%%%%%%%%%%%%%%%%%%%%

Este foi um dos temas principais da minha tese de doutorado.
Meu objetivo era desenvolver um método de inversão que utilizasse a modelagem
direta com tesseroides.
Assim, seria possível utilizar a grande cobertura de dados gravimétricos de
satélites para modelar feições em escala continental levando em
consideração a curvatura da Terra.
A feição que buscamos mapear é a interface entre a crosta e o manto, marcada
pela descontinuidade de Mohorovičić (Moho).

Novamente, minha inspiração para este método surgiu a partir do Prof. João B.
C. Silva.
Em \citet{silva2014}, ele e seus colaboradores aprimoraram o método de
inversão não-linear de \citet{bott}.
Este método estima a profundidade do contato entre bacias sedimentares e o
embasamento cristalino a partir de dados gravimétricos.
A principal vantagem do método de Bott é ser computacionalmente eficiente.
\citet{silva2014} demonstraram que o método de Bott pode ser considerado como
um caso particular do método de optimização Gauss-Newton.
A aproximação consiste em utilizar uma matriz diagonal e linear no lugar da
matriz de sensibilidade (ou Jacobiana).
A principal desvantagem do método é sua sensibilidade à presença de ruídos
aleatórios nos dados.

Baseado na formulação de \citet{silva2014} para o método de Bott, desenvolvi um
método para a inversão não-linear de dados gravimétricos para estimar o relevo
da Moho na América do Sul.
Utilizei a regularização de suavidade para estabilizar a inversão.
Porém, fui capaz de reter a eficiência computacional do método de Bott
utilizando matrizes esparsas na implementação computacional do método.
Também desenvolvi uma técnica de validação utilizando observações sismológicas
da profundidade da Moho para estimar o contraste de densidade e a profundidade
da Moho de referência.

Publiquei este método de inversão e os resultados que obtivemos para a América
do Sul no artigo:

\begin{displayquote}
    UIEDA, L.; BARBOSA, V. C. F. Fast nonlinear gravity inversion in spherical
    coordinates with application to the South American Moho. Geophysical
    Journal International , v. 208, p. 162-176, 2017.\footnote{Código fonte
    disponível em
    \url{https://github.com/pinga-lab/paper-moho-inversion-tesseroids}}
\end{displayquote}

O modelo que criei para a profundidade da Moho na América do Sul está
disponível gratuitamente em \url{https://doi.org/10.6084/m9.figshare.3987267}
sob uma licença \textit{Creative Commons}.
Até o presente momento (10 de Julho de 2017), o modelo foi
baixado 230 vezes.


\subsection{Tutoriais sobre geofísica}
%%%%%%%%%%%%%%%%%%%%%%%%%%%%%%%%%%%%%%%%%%%%%%%%%%%%%%%%%%%%%%%%%%%%%%%%%%%%%%%

Publiquei dois trabalhos em uma série de tutoriais sobre geofísica iniciada por
Matt Hall da Agile Scientific \citep{hall_user_2016} para a revista \textit{The
Leading Edge}:

\begin{displayquote}
    UIEDA, L. Step-by-step NMO correction. The Leading Edge, v. 36, p.
    179-180, 2017.\footnote{Código fonte disponível em
    \url{https://github.com/pinga-lab/nmo-tutorial}}
\end{displayquote}

\begin{displayquote}
    UIEDA, L.; OLIVEIRA, V. C.; BARBOSA, V. C. F. Geophysical tutorial: Euler
    deconvolution of potential-field data. The Leading Edge, v. 33, p. 448-450,
    2014.\footnote{Código fonte disponível em
    \url{https://github.com/pinga-lab/paper-tle-euler-tutorial}}
\end{displayquote}

O objetivo da série de tutoriais é explicar conceitos da geofísica de forma
prática e acessível.
Um dos requisitos para a publicação é a inclusão de código fonte que possa ser
utilizado pelo leitor para explorar os conceitos abordados no tutorial.
O código é disponibilizado com uma licença de software livre
e o texto do tutorial é publicado no formato ``acesso livre'' (\textit{open
access}) com uma licença da \textit{Creative Commons}.

Os 9 artigos da revista \textit{The Leading Edge} mais acessados nos últimos 12
meses são todos pertencentes à série de tutoriais\footnote{Segundo
\url{http://library.seg.org/action/showMostReadArticles?journalCode=leedff}
(acessado em 10 de Julho de 2017).}.
O artigo ``Step-by-step NMO correction'' está no sexto lugar dessa lista.



\subsection{Camada Equivalente}
%%%%%%%%%%%%%%%%%%%%%%%%%%%%%%%%%%%%%%%%%%%%%%%%%%%%%%%%%%%%%%%%%%%%%%%%%%%%%%%

A técnica da camada equivalente é utilizada para o processamento de dados de
métodos potenciais.
Suas aplicações incluem a interpolação de dados, continuação para cima, redução
ao polo, remoção de ruídos aleatórios e cálculo de derivadas.
A camada equivalente é aplicada em dois passos:
primeiro, estimamos os coeficientes de uma série de funções harmônicas que
melhor ajustam os dados observados;
segundo, utilizamos os coeficientes estimados para realizar a transformação
desejada através da modelagem direta.
É comum utilizar como funções harmônicas o efeito de uma malha regular de
fontes pontuais.
Dessa forma, os coeficientes estimados possuem significado físico.
Por exemplo, no caso da gravimetria os coeficientes representam as densidades
de massas pontuais.
A camada equivalente é capaz de operar em dados distribuídos de forma irregular
e é geralmente mais estável que métodos que utilizam a Transformada de Fourier.
Porém, o primeiro passo para sua aplicação é uma inversão, o que a torna
computacionalmente custosa.

Das muitas conversas que tive com o Vanderlei durante a pós-graduação,
surgiu a ideia de parametrizar a camada equivalente utilizando polinômios
bidimensionais no lugar de fontes pontuais.
A camada seria dividida em janelas e a distribuição de propriedades físicas
dentro de cada janela seria representada por um polinômio.
Essa parametrização nos possibilitaria estimar os coeficientes desses
polinômios ao invés de estimar os valores de propriedade física de cada fonte
pontual.
Esta mudança reduz drasticamente o número de parâmetros a serem estimados na
inversão.

Essa ideia foi executada pelo Vanderlei e se tornou parte de sua tese de
doutorado.
Publicamos o método desenvolvido
em um trabalho completo nos anais do V Simpósio Brasileiro de Geofísica em 2012
e na revista \textit{Geophysics} em 2013:

\begin{displayquote}
    OLIVEIRA JR., V. C.; BARBOSA, V. C. F.; UIEDA, L.
    Camada Equivalente Polinomial. In: V Simpósio Brasileiro de Geofísica,
    2012.
\end{displayquote}

\begin{displayquote}
    OLIVEIRA Jr., V. C.; BARBOSA,  V. C. F.; UIEDA, L. Polynomial equivalent
    layer. Geophysics, 78(1), G1-G13, 2013.
\end{displayquote}



\subsection{Estimação da direção de magnetização}
%%%%%%%%%%%%%%%%%%%%%%%%%%%%%%%%%%%%%%%%%%%%%%%%%%%%%%%%%%%%%%%%%%%%%%%%%%%%%%%

Frequentemente, a interpretação e inversão de dados de anomalia magnética de
campo total requer o conhecimento da direção de magnetização da fonte.
Técnicas como a redução ao polo e a inversão 3D não-linear não podem ser
aplicadas sem se ter conhecimento da direção da magnetização total do corpo
geológico.
A dissertação de mestrado da aluna Daiana P. Sales, que foi orientada pelo
Vanderlei, propõe uma forma de estimar a direção de magnetização de corpos
aproximadamente esféricos a partir da anomalia de campo total.
Acompanhei o trabalho da Daiana e do Vanderlei de perto e auxiliei com a
elaboração de testes de sensibilidade e a implementação do método em linguagem
Python como parte do \textit{Fatiando a Terra}.
Publicamos os resultados decorrentes da dissertação da Daiana no artigo:

\begin{displayquote}
    OLIVEIRA JR., V. C.; SALES, D. P.; BARBOSA, V. C. F.; UIEDA, L. Estimation
    of the total magnetization direction of approximately spherical bodies.
    Nonlinear Processes in Geophysics, v. 22, p. 215-232, 2015.\footnote{Código
    fonte disponível em
    \url{https://github.com/pinga-lab/Total-magnetization-of-spherical-bodies}}
\end{displayquote}



\subsection{Deconvolução de Euler}
%%%%%%%%%%%%%%%%%%%%%%%%%%%%%%%%%%%%%%%%%%%%%%%%%%%%%%%%%%%%%%%%%%%%%%%%%%%%%%%

A Deconvolução de Euler (DE) \citep{thompson1982,reid1990} é uma metodologia
muito utilizada na interpretação de dados magnéticos.
A DE estima a posição de fontes ideais (esferas, planos, linhas, etc)
a partir da anomalia magnética e de suas derivadas nas direções x, y e z.
A grande popularidade da DE se deve ao fato de ela ser relativamente simples de
se programar e de possuir um baixo custo computacional.
Um dos desafios encontrados durante sua utilização é a quantidade excessiva de
soluções geradas pelo método.
Muitas dessas soluções são erradas e devem ser descartadas.

\citet{barbosa1999} demostrou que as soluções produzidas para as coordenadas x,
y e z da fonte formam patamares ao longo do valor correto de cada coordenada.
O objetivo da dissertação de mestrado do aluno Felipe F. Melo, orientado pela
Profa. Valéria, foi desenvolver um método que utilizasse esses patamares para
eliminar as soluções esporádicas da Deconvolução de Euler.
Participei desse trabalho durante a elaboração da metodologia e durante a
escrita do artigo:

\begin{displayquote}
    MELO, F. F.; BARBOSA, V. C. F.; UIEDA, L.; OLIVEIRA Jr, V. C.; SILVA, J. B.
    C.  Estimating the nature and the horizontal and vertical positions of 3D
    magnetic sources using Euler deconvolution. Geophysics, v. 78, p. J87-J98,
    2013.
\end{displayquote}


%%%%%%%%%%%%%%%%%%%%%%%%%%%%%%%%%%%%%%%%%%%%%%%%%%%%%%%%%%%%%%%%%%%%%%%%%%%%%%%%
\section{Ensino}

Assim como muitos outros professores universitários, eu nunca recebi treinando
formal em técnicas de ensino.
Por isso, busquei ler o máximo que pude a respeito desse assunto para me
familiarizar com o seu estado da arte.
Uma tendência que observei com frequência é a mudança para um estilo de ensino
envolvendo menos palestras e mais atividades práticas.
Alguns educadores levam essa prática ao extremo, dedicando $100\%$ do tempo em
sala de aula a exercícios e atividades práticas.
Outra tendência comum é o uso de técnicas de avaliação formativa, que são
aplicadas de forma contínua durante o curso e têm função diagnóstica.
Por exemplo, um professor pode utilizar questões de múltipla escolha para
rapidamente avaliar quais são as falhas no entendimento dos alunos.
Dessa forma, o professor é capaz de adaptar sua aula para sanar essas dúvidas.
Busco sempre incorporar essas técnicas nas minhas aulas e aprimorar minhas
habilidades didáticas.
Utilizo a programação para criar um material didático interativo para as minhas
aulas práticas, muitas vezes utilizando as funções existentes no
\textit{Fatiando a Terra}.
Além disso, coleto a opinião dos alunos de forma anônima no final de cada
curso.
Essas avaliações me permitem determinar o que devo manter como parte do curso e
o que pode ser feito para melhorá-lo.
Assim como todo o resto da minha produção, disponibilizo o material didático
que desenvolvo através da minha página pessoal
\url{http://www.leouieda.com/teaching}.


\subsection{Cursos de curta duração}

Minha primeira experiência como instrutor foi quando ministrei o minicurso
``Tópicos de inversão em geofísica'' junto com o Prof. Dr. Vanderlei C.
Oliveira Jr. na XVI Escola de Verão de Geofísica do
IAG/USP\footnote{\url{https://github.com/pinga-lab/inversao-iag-2012}}.
Na época, éramos alunos de doutorado no Observatório Nacional.
Optamos por utilizar uma abordagem prática para as aulas e utilizamos o
\textit{Fatiando a Terra} nos exercícios e exemplos.
O curso foi bem recebido pelos alunos, muitos dos quais fizeram comentários
positivos a respeito das atividades práticas.
A apostila que escrevemos para esse curso está disponível livremente em
\url{https://doi.org/10.6084/m9.figshare.1192984.v3}.
Atualmente (10 de Julho de 2017), a apostila foi baixada mais de 400 vezes.

Depois dessa experiência eu tive certeza de que ensinar é algo que eu gostaria
de fazer como minha carreira.
Após assumir o cargo de Professor na UERJ, ministrei novamente o minicurso de
inversão na Universidade de Brasília, dessa vez
sozinho\footnote{\url{https://github.com/pinga-lab/inversao-unb-2014}}.
Reutilizei grande parte do material original que desenvolvemos, porém adaptado
para um curso de menor duração.

Em 2016, ministrei dois minicursos sobre programação em Python com ênfase em
ciências da Terra.
O primeiro foi ``Python como uma ferramenta numérica em Ciências da
Terra: Uma nova abordagem de
programação''\footnote{\url{https://github.com/leouieda/verao2016}}, ministrado
durante
a XVIII Escola de Verão de Geofísica do IAG/USP em conjunto com os Profs. Drs.
Marcelo Bianchi e Victor Sacek.
Nesse curso, ensinei os conceitos básicos da utilização do \textit{Fatiando a
Terra}.
O segundo foi ``Python para
Geologia''\footnote{\url{https://github.com/leouieda/python-geologia-2016}},
ministrado durante a
VII Semana Acadêmica de Geologia da UERJ.
Objetivo de ambos os cursos era fornecer o conhecimento mínimo necessário para
que os alunos pudessem começar a estudar a linguagem Python por conta própria.
Em 2017, ministrei o minicurso ``Introduction to Python
Workshop''\footnote{\url{https://github.com/leouieda/python-hawaii-2017}} na
University of Hawaii adaptando o material desenvolvido para o curso da UERJ.
Escrevi a respeito desse curso, dos métodos de ensino utilizados e das
avaliações que recebi dos alunos em
\url{http://www.leouieda.com/blog/python-hawaii-2017.html}.


\subsection{Disciplinas de graduação}

Desde minha contratação na UERJ em 2014, ministrei as seguintes disciplinas:

\begin{itemize}
    \item \textit{Geologia Geral 1} e \textit{Matemática Especial
        I}\footnote{\url{https://github.com/mat-esp}} para o curso de
        Oceanografia;
    \item \textit{Mineralogia e Petrologia} para o curso de Biologia;
    \item \textit{Geofísica
        1}\footnote{\url{https://github.com/leouieda/geofisica1}} e
        \textit{Geofísica
        2}\footnote{\url{https://github.com/leouieda/geofisica2}} para o curso
        de Geologia;
\end{itemize}

A disciplina Matemática Especial I consiste em uma breve introdução à
programação seguida dos conceitos básicos de cálculo numérico.
Os alunos passam cerca de $80\%$ do tempo em sala de aula trabalhando em grupos
para a solução de exercícios.
Para cada tema abordado, os grupos recebem repositórios individuais na página
\url{https://www.github.com} contendo as instruções para os exercícios.
A entrega dos exercícios é feita de forma digital através desses repositórios
que são agrupados em contas de usuário que eu controlo.
Por exemplo, todas as soluções entregues pelos grupos de 2015 estão na página
\url{https://github.com/mat-esp-2015}.
Foi somente através dessa abordagem digital que eu fui capaz de administrar as
duas turmas, com aproximadamente 40 alunos em cada uma, do segundo semestre de
2015.


As duas disciplinas de geofísica do curso de geologia são divididas em aulas
teóricas e aulas práticas em proporções iguais.
Em ambas disciplinas, utilizo o \textit{Fatiando a Terra} em conjunto com
documentos chamados \textit{Jupyter
notebooks}\footnote{\url{http://jupyter.org}} para criar os exercícios das
aulas práticas.
Os \textit{notebooks} permitem inserir código, texto, imagens, equações e
vídeos em único documento interativo.
Os alunos trabalham na atividade prática em grupos e são guiados pelo texto dos
\textit{notebooks}, que contém explicações e perguntas.
Em 2016, a primeira turma de alunos que cursou as disciplinas de geofísica
comigo me escolheu como Paraninfo em sua formatura.



\subsection{Orientações, coorientações e treinamento}

Atualmente, sou coorientador da tese de doutorado do aluno Santiago Soler, cujo
orientador é o Prof. Dr. Mario Ernesto Gimenez da Universidad Nacional de San
Juan, Argentina.
Conheci o Santigo através das contribuições que ele submeteu para o
\textit{Fatiando a Terra}.
Em 2016, ele e o Prof. Gimenez me convidaram para ser coorientador no projeto
de doutorado ``Modelos de inversão conjunta de dados gravimétricos e de função
do receptor através do uso de tesseroides''.
O trabalho está em estágio inicial e ainda não possui resultados publicados.

Em 2014, fui selecionado pela UERJ para participar do projeto QUALITEC.
Esse projeto financiava bolsas para técnicos de nível superior que trabalhariam
nos laboratórios da UERJ e seriam treinados nas suas respectivas áreas de
atuação.
Recebi uma dessas bolsas para o Laboratório de Geofísica de Exploração, do qual
sou o coordenador.
Selecionei para a vaga o técnico Victor Thadeu Xavier de Almeida que é
extremamente competente e tem formação sólida em física e geofísica.
Como parte do programa QUALITEC, treinei o Victor na linguagem de programação
Python e nos diversas técnicas do desenvolvimento de software.
Ele fez contribuições para o \textit{Fatiando a Terra} e sua ajuda foi
indispensável nas aulas de Matemática Especial I.

Antes da minha viagem para a University of Hawaii, orientei dois alunos de
iniciação científica.
A aluna Fernanda Vianna Gatts trabalhou no projeto ``Variações do volume do
aquífero Guarani determinadas por dados de gravidade do satélite GRACE''.
O aluno Vinícius Vianna Riguête do curso de Geologia continua desenvolvendo seu
trabalho no projeto ``Estudo gravimétrico das intrusões
Eo-cretácicas da Namíbia''.


%%%%%%%%%%%%%%%%%%%%%%%%%%%%%%%%%%%%%%%%%%%%%%%%%%%%%%%%%%%%%%%%%%%%%%%%%%%%%%%%
\section{Curriculum Vitae}

\subsection{Formação acadêmica}

\begin{itemize}
    \item \textbf{Pós-doutorado} (02/2017 - Presente),
        University of Hawaii, Honolulu, E.U.A.
        Projeto: Expansion of the Generic Mapping Tools (GMT) to the Python
        programming language.
        Supervisor: Paul Wessel.
        Bolsista da National Science Foundation (NSF), E.U.A.
    \item \textbf{Doutorado em Geofísica} (11/2011 - 04/2016),
        Observatório Nacional, Rio de Janeiro, Brasil.
        Tese: Modelagem direta e inversão de campos gravitacionais em
        coordenadas esféricas.
        Orientadora: Valéria Cristina Ferreira Barbosa.
        Bolsista da Coordenação de Aperfeiçoamento de Pessoal de Nível
        Superior (CAPES).
    \item \textbf{Mestrado em Geofísica} (02/2010 - 10/2011),
        Observatório Nacional, Rio de Janeiro, Brasil.
        Dissertação: Robust 3D gravity gradient inversion by planting anomalous
        densities.
        Orientadora: Valéria Cristina Ferreira Barbosa.
        Bolsista da Coordenação de Aperfeiçoamento de Pessoal de Nível
        Superior (CAPES).
    \item \textbf{Intercâmbio Internacional} (08/2008 - 05/2009),
        York University, Toronto, Canadá.
    \item \textbf{Bacharelado em Geofísica} (02/2004 - 12/2009),
        Universidade de São Paulo, São Paulo, Brasil.
        Trabalho de conclusão: Cálculo do tensor gradiente gravimétrico
        utilizando tesseroides.
        Orientadora: Naomi Ussami.
        Bolsista da Sociedade Brasileira de Geofísica (SBGf).
\end{itemize}


\subsection{Atuação profissional}

\begin{itemize}
    \item \textbf{Professor Assistente} (02/2014 - Presente), regime de 40
        horas semanais,
        Universidade do Estado do Rio de Janeiro, Rio de Janeiro, Brasil.
        Coordenador do Laboratório de Geofísica de Exploração. Responsável
        pelas disciplinas Geofísica 1, Geofísica 2 e Matemática Especial I.
\end{itemize}


\subsection{Coordenação de Projetos}

\begin{itemize}
    \item \textbf{Projeto Qualitec 2014 para bolsista de Nível Superior}
        (10/2014 - Presente).
        Bolsa para treinamento de um técnico de nível superior para o
        Laboratório de Geofísica de Exploração (LAGEX).
        Financiador: Universidade do Estado do Rio de Janeiro.
\end{itemize}


\subsection{Revisor de periódicos}

\begin{itemize}
    \item Computers \& Geosciences - Início em 2011.
    \item Geophysics - Início em 2013.
    \item Central European Journal of Geosciences (Open Geosciences) - Início em 2013.
    \item Pure and Applied Geophysics - Início em 2015.
    \item Journal of Applied Geophysics - Início em 2015.
    \item Geophysical Prospecting - Início em 2015.
    \item Geophysical Journal International - Início em 2015.
\end{itemize}

\subsection{Artigos publicados}

\begin{enumerate}
\item \textbf{Uieda, L.}, and V. C. F. Barbosa (2017), Fast nonlinear gravity inversion in spherical coordinates with application to the South American Moho, Geophys. J. Int., 208(1), 162-176, doi:10.1093/gji/ggw390.
\item \textbf{Uieda, L.} (2017), Step-by-step NMO correction, The Leading Edge, 36(2), 179-180, doi:10.1190/tle36020179.1.
\item \textbf{Uieda, L.}, V. Barbosa, and C. Braitenberg (2016), Tesseroids: Forward-modeling gravitational fields in spherical coordinates, GEOPHYSICS, F41-F48, doi:10.1190/geo2015-0204.1.
\item Carlos, D. U., \textbf{L. Uieda}, and V. C. F. Barbosa (2016), How two gravity-gradient inversion methods can be used to reveal different geologic features of ore deposit -- A case study from the Quadrilátero Ferrífero (Brazil), Journal of Applied Geophysics, doi:10.1016/j.jappgeo.2016.04.011.
\item Oliveira Jr., V. C., D. P. Sales, V. C. F. Barbosa, and \textbf{L. Uieda} (2015), Estimation of the total magnetization direction of approximately spherical bodies, Nonlin. Processes Geophys., 22(2), 215-232, doi:10.5194/npg-22-215-2015.
\item \textbf{Uieda, L.}, V. C. Oliveira Jr., and V. C. F. Barbosa (2014), Geophysical tutorial: Euler deconvolution of potential-field data, The Leading Edge, 33(4), 448-450, doi:10.1190/tle33040448.1.
\item Carlos, D. U., \textbf{L. Uieda}, and V. C. F. Barbosa (2014), Imaging iron ore from the Quadrilátero Ferrífero (Brazil) using geophysical inversion and drill hole data, Ore Geology Reviews, 61, 268-285, doi:10.1016/j.oregeorev.2014.02.011.
\item Melo, F. F., V. C. F. Barbosa, \textbf{L. Uieda}, V. C. Oliveira, and J. B. C. Silva (2013), Estimating the nature and the horizontal and vertical positions of 3D magnetic sources using Euler deconvolution, GEOPHYSICS, 78(6), J87-J98, doi:10.1190/geo2012-0515.1.
\item Oliveira Jr., V. C., V. C. F. Barbosa, and \textbf{L. Uieda} (2013), Polynomial equivalent layer, GEOPHYSICS, 78(1), G1-G13, doi:10.1190/geo2012-0196.1.
\item \textbf{Uieda, L.}, and V. C. F. Barbosa (2012), Robust 3D gravity gradient inversion by planting anomalous densities, GEOPHYSICS, 77(4), G55-G66, doi:10.1190/geo2011-0388.1.
\end{enumerate}

\subsection{Trabalhos completos publicados em anais de eventos}

\begin{enumerate}
\item Melo, F. F., V. C. F. Barbosa, \textbf{L. Uieda}, V. C. O. Jr, and J. B. C. Silva (2014), A Single Euler Solution Per Anomaly, in 76th EAGE Conference and Exhibition 2014.
\item \textbf{Uieda, L.}, V. C. Oliveira Jr, and V. C. F. Barbosa (2013), Modeling the Earth with Fatiando a Terra, in Proceedings of the 12th Python in Science Conference, edited by S. van der Walt, J. Millman, and K. Huff, pp. 91-98.
\item \textbf{Uieda, L.}, and V. C. F. Barbosa (2012), Use of the ``shape-of-anomaly'' data misfit in 3D inversion by planting anomalous densities, in SEG Annual Meeting, pp. 1-6, Society of Exploration Geophysicists.
\item Carlos, D. U., \textbf{L. Uieda}, Y. Li, V. C. F. Barbosa, M. A. Braga, G. Angeli, and G. Peres (2012), Iron ore interpretation using gravity-gradient inversions in the Carajás, Brazil, in SEG Annual Meeting, pp. 1-5, Society of Exploration Geophysicists.
\item Oliveira Jr., V. C., V. C. F. Barbosa, and \textbf{L. Uieda} (2012), Camada Equivalente Polinomial, in V Simpósio Brasileiro de Geofísica.
\item \textbf{Uieda, L.}, E. P. Bomfim, C. Braitenberg, and E. Molina (2011), Optimal forward calculation method of the Marussi tensor due to a geologic structure at GOCE height, in Proceedings of the 4th International GOCE User Workshop.
\item \textbf{Uieda, L.}, and V. C. F. Barbosa (2011), Robust 3D gravity gradient inversion by planting anomalous densities, in SEG Annual Meeting.
\item \textbf{Uieda, L.}, and V. C. F. Barbosa (2011), 3D gravity gradient inversion by planting density anomalies, in 73th EAGE Conference and Exhibition incorporating SPE EUROPEC.
\item \textbf{Uieda, L.}, and V. C. Barbosa (2011), 3D gravity inversion by planting anomalous densities, in 12th International Congress of the Brazilian Geophysical Society.
\item Carlos, D. U., \textbf{L. Uieda}, V. C. F. Barbosa, M. A. Braga, and A.  A. S. Gomes (2011), In-depth imaging of an iron orebody from Quadrilatero Ferrifero using 3D gravity gradient inversion, in SEG Annual Meeting.
\item Carlos, D. U., V. C. Barbosa, \textbf{L. Uieda}, and M. A. Braga (2011), Inversão de Dados de Aerogradiometria Gravimétrica 3D-Ftg Aplicada a Exploração Mineral na Região do Quadrilátero Ferrífero, in 12th International Congress of the Brazilian Geophysical Society.
\end{enumerate}


\subsection{Programas de computador}

\begin{itemize}
    \item \textbf{Tesseroids} (2009 - Presente).
        Página oficial: \url{http://tesseroids.leouieda.com}.
        Código fonte: \url{https://github.com/leouieda/tesseroids}.
        Linguagem de programação: C.
        Licença: BSD 3-clause License.
        DOI da versão mais recente (1.2.1):
        \url{https://doi.org/10.5281/zenodo.582366}.
    \item \textbf{Fatiando a Terra} (2010 - Presente).
        Página oficial: \url{http://fatiando.org}.
        Código fonte: \url{https://github.com/fatiando/fatiando}.
        Linguagem de programação: Python.
        Licença: BSD 3-clause License.
        DOI da versão mais recente (0.5):
        \url{https://doi.org/10.5281/zenodo.157746}.
    \item \textbf{GMT/Python} (2017 - Presente).
        Página oficial: \url{https://genericmappingtools.github.io/gmt-python/}.
        Código fonte: \url{https://github.com/GenericMappingTools/gmt-python}.
        Linguagem de programação: Python.
        Licença: BSD 3-clause License.
\end{itemize}

\subsection{Apresentações de trabalho}

\begin{enumerate}
\item \textbf{Uieda, L.} and P. Wessel (2017), Bringing the Generic Mapping
    Tools to Python. Scientific Computing with Python (Scipy). Austin, E.U.A.
\item \textbf{Uieda, L.} (2017), Inverting gravity to map the Moho: A new
    method and the open source software that made it possible. Department of
    Geology and Geophysics, University of Hawaii, Honolulu, E.U.A.
\item \textbf{Uieda, L.} (2016), Fatiando a Terra: Construindo uma base para
    ensino e pesquisa de geofísica. Observatório Nacional, Rio de Janeiro,
    Brasil.
\item \textbf{Uieda, L.} (2015), Fatiando a Terra: Construindo uma base para
    ensino e pesquisa de geofísica. Instituto de Astronomia, Geofísica e
    Ciências Atmosféricas, Universidade de São Paulo, São Paulo, Brasil.
\item \textbf{Uieda, L.} and Barbosa, V. C. F. (2014), Gravity inversion in
    spherical coordinates using tesseroids. EGU General Assembly, Viena,
    Áustria.
\item \textbf{Uieda, L.}, Oliveira Jr., V. C. and Barbosa, V. C. F.  (2014),
    Using Fatiando a Terra to solve inverse problems in geophysics. Scientific
    Computing with Python (Scipy). Austin, E.U.A.
\item \textbf{Uieda, L.} and Barbosa, V. C. F. (2013), 3D magnetic inversion by
    planting anomalous densities. AGU Meeting of the Americas, Cancun, México.
\item \textbf{Uieda, L.}, Oliveira Jr., V. C., and Barbosa, V. C. F.  (2013),
    Modeling the Earth with Fatiando a Terra. Scientific Computing with Python
    (Scipy). Austin, E.U.A.
\item \textbf{Uieda, L.} and Barbosa, V. C. F. (2012), Rapid 3D inversion of
    gravity and gravity gradient data to test geologic hypotheses.
    International Symposium on Gravity, Geoid and Height Systems, Veneza,
    Itália.
\item \textbf{Uieda, L.} and Barbosa, V. C. F. (2012), Use of the
    ``shape-of-anomaly'' data misfit in 3D inversion by planting anomalous
    densities. SEG Annual Meeting, Las Vegas, E.U.A.
\item Carlos, D. U., \textbf{Uieda, L.}, Li, Y., Barbosa, V. C. F., Braga, M.
    A., Angeli, G., Peres, G. (2012), Iron ore interpretation using
    gravity-gradient inversions in the Carajás, Brazil. SEG Annual Meeting, Las
    Vegas, E.U.A.
\item \textbf{Uieda, L.} and Barbosa, V. C. F. (2011), 3D gravity gradient
    inversion by planting density anomalies. EAGE Conference and Exhibition,
    Viena, Áustria.
\item \textbf{Uieda, L.} and Barbosa, V. C. F. (2011), Robust 3D gravity
    gradient inversion by planting anomalous densities. SEG Annual Meeting, San
    Antonio, E.U.A.
\item \textbf{Uieda, L.} and Barbosa, V. C. F. (2011), 3D gravity inversion by
    planting anomalous densities. International Congress of the Brazilian
    Geophysical Society, Rio de Janeiro, Brasil.
\item \textbf{Uieda, L.}, Ussami, N., and Braitenberg, C. (2010), Computation
    of the gravity gradient tensor due to topographic masses using tesseroids.
    AGU Meeting of the Americas, Foz do Iguaçu, Brasil.
\end{enumerate}


\subsection{Prêmios e títulos}

\begin{itemize}
\item PACE student travel grant para participação no 73rd EAGE Conference and
    Exhibition 2011, Vienna, Austria, financiado pela European Association of
    Geoscientists and Engineers.
\item Student travel grant para particiação no SEG Annual Meeting 2011, San
    Antornio, E.U.A., financiado pela Near Surface Geophysics Section (NSGS).
\end{itemize}


\subsection{Participações em bancas}

\begin{itemize}
\item Menezes, P. T. L., \textbf{Uieda, L.}, Santos, L. A. (2016).
    Participação em banca de Natacha Medeiros Rocha. Mestrado em Análise de
    Bacias e Faixas Móveis, Universidade do Estado do Rio de Janeiro.
\item Menezes, P. T. L., Mane, M. A., \textbf{Uieda, L.} (2016).  Participação
    em banca de Henrique Cavalcanti Pequeno. Trabalho de Conclusão de Curso
    (Graduação em Geologia), Universidade do Estado do Rio de Janeiro.
\end{itemize}


\subsection{Cursos de curta duração ministrados}

\begin{itemize}
    \item \textbf{Uieda, L.} (2017). Introduction to Python Workshop.
        University of Hawaii, E.U.A.
    \item \textbf{Uieda, L.}, Bianch, M., Sacek, V. (2016).
        Python como uma ferramenta numérica em ciências da Terra: Uma nova
        abordagem de programação. Universidade de São Paulo.
    \item \textbf{Uieda, L.} (2014). Python para Geologia. Universidade do
        Estado do Rio de Janeiro.
    \item \textbf{Uieda, L.} (2014). Tópicos de inversão em geofísica.
        Universidade de Brasília.
    \item Oliveira Jr., V. C., \textbf{Uieda, L.} (2012). Tópicos de inversão
        em geofísica.  Universidade de São Paulo.
\end{itemize}


%%%%%%%%%%%%%%%%%%%%%%%%%%%%%%%%%%%%%%%%%%%%%%%%%%%%%%%%%%%%%%%%%%%%%%%%%%%%%%%%
\bibliographystyle{gji}
\bibliography{references}
\addcontentsline{toc}{section}{Referências}

\end{document}
