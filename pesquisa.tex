\section{Pesquisa}

Minha pesquisa se concentra na área de problemas inversos em métodos
potenciais.
Geralmente, meus trabalhos são avanços metodológicos e são acompanhados por
um código fonte que os implementa.
Como mencionei no capítulo \ref{software}, acredito que o código fonte que
acompanha uma publicação é tão importante quanto a descrição de sua
metodologia.
Muitas vezes é impossível reproduzir os resultados de um trabalho
sem ter acesso ao software que os gerou.
Logo, é crucial que o código esteja disponível para ser revisado pela
comunidade científica.
Para tanto, disponibilizo o código e dados (à medida do possível) necessários
para reproduzir os resultados de meus trabalhos como primeiro autor.
Cada trabalho é acompanhado de um repositório na página do Grupo de Pesquisa em
Problemas Inversos em Geofísica (\url{https://github.com/pinga-lab}).
Cada repositório também é arquivado permanentemente e recebe um Digital Object
Identifier (DOI) através de serviços como Zenodo (\url{http://zenodo.org}) e
figshare (\url{https://figshare.com}).

A seguir, apresento uma reflexões sobre os aspectos de minha formação que me
levaram a essa área de pesquisa e sobre os diferentes trabalhos que formam
minha produção acadêmica.


\subsection{Formação}

Desde o início da graduação me senti intrigado pelos métodos de inversão.
Sempre ouvia de alunos veteranos, ou até mesmo de professores, que esse era um
assunto extremamente complexo.
Conhecendo minha personalidade, creio que meu interesse inicial sobre o assunto
era puramente devido ao desafio.
Por sorte, a disciplina que forma a base do conhecimento de problemas inversos,
a álgebra linear, também foi um tema que despertou meu interesse.
Por outro lado, devo confessar que, a princípio, os métodos potenciais não me
interessaram tanto quando a inversão.
Minha primeira impressão da gravimetria e magnetometria foi que eram métodos
simples.
Porém, terminei minha primeira disciplina sobre o assunto completamente confuso
e com mais dúvidas do que tinha antes de cursá-la.
Hoje percebo que essas são características de um assunto complexo e
interessante e de uma aula de qualidade.

Fiz minha primeira iniciação científica no laboratório de paleomagnetismo com
bolsa da FAPESP e orientação do Prof. Manoel S. D'Agrella Filho.
Ao final do período de um ano da bolsa, decidi não continuar nessa área.
Em seguida, busquei outro projeto que unisse a geofísica
com meu interesse pela programação.
Em 2007, iniciei o projeto sob orientação da
Profa. Naomi Ussami que resultou no software \textit{Tesseroids}.

Em 2008, realizei um programa de intercâmbio de 10 meses na York University,
Canadá.
Parte do motivo para essa escolha é a excelência do país nas áreas de
gravimetria e geodésia.
Lá, cursei disciplinas sobre geodésia física, gravimetria e ajuste de redes
através do método dos mínimos quadrados.

Em 2010, ingressei no Mestrado em Geofísica do Observatório Nacional sob
orientação da Profa. Valéria C. F. Barbosa.
Meu projeto era criar um método de inversão 3D de dados de gradiometria
gravimétrica, tema que contava com atenção internacional e era pioneiro no
cenário nacional.
Nesse ponto, percebi que as disciplinas cursadas na York me forneceram a base
necessária para compreender a inversão e a gravimetria com muito mais
facilidade.
Aprendi com a Profa. Valéria
as diferentes vertentes e sutilezas da inversão,
a arte da elaboração de artigo científico
e sua ética profissional impecável.
Também devo muito do meu aprendizado durante a pós-graduação às longas
conversas e debates com meu amigo Vanderlei C. Oliveira Jr. (atualmente
pesquisador do Observatório Nacional).



\subsection{Modelagem direta de campos gravitacionais com tesseroides}

Comecei a desenvolver esse assunto durante minha iniciação científica com a
Profa. Naomi de 2007 a 2009.
A princípio, meu trabalho simplesmente aprender a metodologia de
\citet{wild-pfeiffer2008} e transformá-la em um programa de computador.
Apresentei meu primeiro trabalho em um evento internacional sobre os resultados
de meu trabalho de conclusão de curso:

\begin{displayquote}
    UIEDA, L.; USSAMI, N. ; BRAITENBERG, C. . Computation of the gravity
    gradient tensor due to topographic masses using tesseroids. In: AGU Meeting
    of the Americas, 2010, Foz do Iguaçu. Eos Trans. AGU, Meet. Am. Suppl.,
    Abstract G22A-04, 2010. v. 91.

    (Apresentação disponível em
    \url{http://www.leouieda.com/talks/agu2010.html})
\end{displayquote}

Retomei esse projeto em 2011 durante minha estadia em Trieste com a Profa.
Braitenberg para atualizar o software e a metodologia para o cálculo dos campos
gravitacionais de um tesseroide (prisma esférico).
Utilizei as equações otimizadas de \citet{grombein2013} para eliminar os
problemas com singularidades da formulação de \citet{wild-pfeiffer2008}.
Também desenvolvi um método de discretização adaptativa dos tesseroides para
garantir a acurácia da integração numérica com a Quadratura Gauss-Legendre.
Sem meu conhecimento, um algoritmo similar \citep{li2011} havia sido publicado
no mesmo período em que estava em Trieste, inviabilizando a nossa publicação do
método.
Ainda em Trieste, busquei sem sucesso caracterizar o erro numérico envolvido
nos cálculos para melhorar controlar a discretização adaptativa.
Publiquei meus esforços com o software \textit{Tesseroids} e os resultados da
análise do erro da quadratura nos anais do 4th International GOCE User
Workshop:

\begin{displayquote}
    UIEDA, L.; BOMFIM, E. P. ; BRAITENBERG, C. ; MOLINA, E. C. . Optimal
    forward calculation method of the Marussi tensor due to a geologic
    structure at GOCE height. In: 4th International GOCE User Workshop, 2011,
    Munich. 4th International GOCE User Workshop, 2011. v. 1. p. 1-5.

    (Pôster e texto disponíveis em
    \url{http://www.leouieda.com/posters/goce2011.html})
\end{displayquote}

Para minha tese de doutorado, iria utilizar a modelagem direta com tesseroides
para desenvolver um método de inversão em coordenadas esféricas.
Porém, para que a inversão seja correta é necessário que a modelagem direta
seja o mais precisa possível.
Logo, continuei com o desenvolvimento do algoritmo de discretização adaptativa
e com os experimentos para caracterizar o erro da quadratura.
Após diversas tentativas frustadas, cheguei aos resultados apresentados no
artigo:

\begin{displayquote}
    UIEDA, LEONARDO; BARBOSA, VALÉRIA C. F. ; BRAITENBERG, CARLA . Tesseroids:
    Forward-modeling gravitational fields in spherical coordinates. Geophysics,
    v. 81, p. F41-F48, 2016.

    (Código fonte disponível em
    \url{https://github.com/pinga-lab/paper-tesseroids})
\end{displayquote}

Neste trabalho, aprimoramos o algoritmo de discretização adaptativa proposto
por \citet{li2011}.
Também determinamos empiricamente valores para o parâmetro
\textit{distance-size ratio}, que controla a discretização adaptativa,
para manter o erro de integração abaixo de $0.1\%$.

Os trabalhos relacionados à essa metodologia estão entre meus trabalhos
citados.
Acredito que isso é devido em partes à disponibilização do software
\textit{Tesseroids} que implementa essa metodologia.
Outro fruto dessa pesquisa é minha coorientação da tese de doutorado do aluno
Santiago Soler com o Prof. Dr. Mario Ernesto Gimenez da Universidad Nacional de
San Juan, Argentina.
A tese de Santiago dará continuidade à modelagem direta com tesseroides,
expandindo a formulação para incluir tesseroides com distribuições internas de
densidades variáveis.


\subsection{Inversão 3D utilizando o método de plantação}

Sementes
Colaboração com Dio
Congressos
SEG e EAGE 2011 travel grant.


\subsection{Inversão 3D em coordenadas esféricas}


Começou com o último trabalho da sementes (EGU 2014).
Inversão da Moho


\subsection{Tutoriais sobre geofísica}

Tutorial Euler
Tutorial NMO


\subsection{Colaborações}


\subsubsection{Aplicações do método de plantação}

Colaborações com Dio.


\subsubsection{Camada Equivalente}

Colaborações com Biroca: PEL,


\subsubsection{Estimação da direção de magnetização}

magdir


\subsubsection{Deconvolução de Euler}

Colaboração com Figura.
